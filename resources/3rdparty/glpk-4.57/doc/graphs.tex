%* graphs.tex *%

%***********************************************************************
%  This code is part of GLPK (GNU Linear Programming Kit).
%
%  Copyright (C) 2000, 2001, 2002, 2003, 2004, 2005, 2006, 2007, 2008,
%  2009, 2010, 2011, 2013 Andrew Makhorin, Department for Applied
%  Informatics, Moscow Aviation Institute, Moscow, Russia. All rights
%  reserved. E-mail: <mao@gnu.org>.
%
%  GLPK is free software: you can redistribute it and/or modify it
%  under the terms of the GNU General Public License as published by
%  the Free Software Foundation, either version 3 of the License, or
%  (at your option) any later version.
%
%  GLPK is distributed in the hope that it will be useful, but WITHOUT
%  ANY WARRANTY; without even the implied warranty of MERCHANTABILITY
%  or FITNESS FOR A PARTICULAR PURPOSE. See the GNU General Public
%  License for more details.
%
%  You should have received a copy of the GNU General Public License
%  along with GLPK. If not, see <http://www.gnu.org/licenses/>.
%***********************************************************************

\documentclass[11pt]{report}
\usepackage{amssymb}
\usepackage[dvipdfm,linktocpage,colorlinks,linkcolor=blue,
urlcolor=blue]{hyperref}
\usepackage{indentfirst}
\usepackage[all]{xy}

\setlength{\textwidth}{6.5in}
\setlength{\textheight}{8.5in}
\setlength{\oddsidemargin}{0in}
\setlength{\topmargin}{0in}
\setlength{\headheight}{0in}
\setlength{\headsep}{0in}
\setlength{\footskip}{0.5in}
\setlength{\parindent}{16pt}
\setlength{\parskip}{5pt}
\setlength{\topsep}{0pt}
\setlength{\partopsep}{0pt}
\setlength{\itemsep}{\parskip}
\setlength{\parsep}{0pt}
\setlength{\leftmargini}{\parindent}
\renewcommand{\labelitemi}{---}

\def\para#1{\noindent{\bf#1}}
\def\synopsis{\para{Synopsis}}
\def\description{\para{Description}}
\def\returns{\para{Returns}}

\renewcommand\contentsname{\sf\bfseries Contents}
\renewcommand\chaptername{\sf\bfseries Chapter}
\renewcommand\appendixname{\sf\bfseries Appendix}

\newenvironment{retlist}
{  \def\arraystretch{1.5}
   \noindent
   \begin{tabular}{@{}p{1in}@{}p{5.5in}@{}}
}
{\end{tabular}}

\begin{document}

\thispagestyle{empty}

\begin{center}

\vspace*{1.5in}

\begin{huge}
\sf\bfseries GNU Linear Programming Kit
\end{huge}

\vspace{0.5in}

\begin{LARGE}
\sf Graph and Network Routines
\end{LARGE}

\vspace{0.5in}

\begin{LARGE}
\sf for GLPK Version 4.49
\end{LARGE}

\vspace{0.5in}
\begin{Large}
\sf (DRAFT, April 2013)
\end{Large}
\end{center}

\newpage

\vspace*{1in}

\vfill

\noindent
The GLPK package is part of the GNU Project released under the aegis of
GNU.

\noindent
Copyright \copyright{} 2000, 2001, 2002, 2003, 2004, 2005, 2006, 2007,
2008, 2009, 2010, 2011, 2013 Andrew Makhorin, Department for Applied
Informatics, Moscow Aviation Institute, Moscow, Russia. All rights
reserved.

\noindent
Free Software Foundation, Inc., 51 Franklin St, Fifth Floor, Boston, MA
02110-1301, USA.

\noindent
Permission is granted to make and distribute verbatim copies of this
manual provided the copyright notice and this permission notice are
preserved on all copies.

\noindent
Permission is granted to copy and distribute modified versions of this
manual under the conditions for verbatim copying, provided also that the
entire resulting derived work is distributed under the terms of
a permission notice identical to this one.

\noindent
Permission is granted to copy and distribute translations of this manual
into another language, under the above conditions for modified versions.

%%%%%%%%%%%%%%%%%%%%%%%%%%%%%%%%%%%%%%%%%%%%%%%%%%%%%%%%%%%%%%%%%%%%%%%%

\newpage

{\setlength{\parskip}{0pt}
\tableofcontents
}

%%%%%%%%%%%%%%%%%%%%%%%%%%%%%%%%%%%%%%%%%%%%%%%%%%%%%%%%%%%%%%%%%%%%%%%%

\chapter{Basic Graph API Routines}

\section{Graph program object}

In GLPK the base program object used to represent graphs and networks
is a directed graph (digraph).

Formally, {\it digraph} (or simply, {\it graph}) is a pair $G=(V,A)$,
where $V$ is a set of {\it vertices}, and $A$ is a set
{\it arcs}.\footnote{$A$ may be a multiset.} Each arc $a\in A$ is an
ordered pair of vertices $a=(x,y)$, where $x\in V$ is called {\it tail
vertex} of arc $a$, and $y\in V$ is called its {\it head vertex}.

Representation of a graph in the program includes three structs defined
by typedef in the header \verb|glpk.h|:

\vspace*{-8pt}

\begin{itemize}
\item \verb|glp_graph|, which represents the graph in a whole,

\item \verb|glp_vertex|, which represents a vertex of the graph, and

\item \verb|glp_arc|, which represents an arc of the graph.
\end{itemize}

\vspace*{-8pt}

All these three structs are ``semi-opaque'', i.e. the application
program can directly access their fields through pointers, however,
changing the fields directly is not allowed --- all changes should be
performed only with appropriate GLPK API routines.

\newenvironment{comment}
{\addtolength{\leftskip}{16pt}\noindent}
{\par\addtolength{\leftskip}{-16pt}}

\para{\bf glp\_graph.} The struct \verb|glp_graph| has the following
fields available to the application program:

\noindent
\verb|char *name;|

\begin{comment}Symbolic name assigned to the graph. It is a pointer to
a null terminated character string of length from 1 to 255 characters.
If no name is assigned to the graph, this field contains \verb|NULL|.
\end{comment}

\noindent
\verb|int nv;|

\begin{comment}The number of vertices in the graph, $nv\geq 0$.
\end{comment}

\noindent
\verb|int na;|

\begin{comment}The number of arcs in the graph, $na\geq 0$.
\end{comment}

\newpage

\noindent
\verb|glp_vertex **v;|

\begin{comment}Pointer to an array containing the list of vertices.
Element $v[0]$ is not used. Element $v[i]$, $1\leq i\leq nv$, is a
pointer to $i$-th vertex of the graph. Note that on adding new vertices
to the graph the field $v$ may be altered due to reallocation. However,
pointers $v[i]$ are not changed while corresponding vertices exist in
the graph.
\end{comment}

\noindent
\verb|int v_size;|

\begin{comment}Size of vertex data blocks, in bytes,
$0\leq v\_size\leq 256$. (See also the field \verb|data| in the struct
\verb|glp_vertex|.)
\end{comment}

\noindent
\verb|int a_size;|

\begin{comment}Size of arc data blocks, in bytes,
$0\leq v\_size\leq 256$. (See also the field \verb|data| in the struct
\verb|glp_arc|.)
\end{comment}

\para{\bf glp\_vertex.} The struct \verb|glp_vertex| has the following
fields available to the application program:

\noindent
\verb|int i;|

\begin{comment}Ordinal number of the vertex, $1\leq i\leq nv$. Note
that element $v[i]$ in the struct \verb|glp_graph| points to the vertex,
whose ordinal number is $i$.
\end{comment}

\noindent
\verb|char *name;|

\begin{comment}Symbolic name assigned to the vertex. It is a pointer to
a null terminated character string of length from 1 to 255 characters.
If no name is assigned to the vertex, this field contains \verb|NULL|.
\end{comment}

\noindent
\verb|void *data;|

\begin{comment}Pointer to a data block associated with the vertex. This
data block is automatically allocated on creating a new vertex and freed
on deleting the vertex. If $v\_size=0$, the block is not allocated, and
this field contains \verb|NULL|.
\end{comment}

\noindent
\verb|void *temp;|

\begin{comment}Working pointer, which may be used freely for any
purposes. The application program can change this field directly.
\end{comment}

\noindent
\verb|glp_arc *in;|

\begin{comment}Pointer to the (unordered) list of incoming arcs. If the
vertex has no incoming arcs, this field contains \verb|NULL|.
\end{comment}

\noindent
\verb|glp_arc *out;|

\begin{comment}Pointer to the (unordered) list of outgoing arcs. If the
vertex has no outgoing arcs, this field contains \verb|NULL|.
\end{comment}

\para{\bf glp\_arc.} The struct \verb|glp_arc| has the following fields
available to the application program:

\noindent
\verb|glp_vertex *tail;|

\begin{comment}Pointer to a vertex, which is tail endpoint of the arc.
\end{comment}

\noindent
\verb|glp_vertex *head;|

\begin{comment}Pointer to a vertex, which is head endpoint of the arc.
\end{comment}

\newpage

\noindent
\verb|void *data;|

\begin{comment}Pointer to a data block associated with the arc. This
data block is automatically allocated on creating a new arc and freed on
deleting the arc. If $v\_size=0$, the block is not allocated, and this
field contains \verb|NULL|.
\end{comment}

\noindent
\verb|void *temp;|

\begin{comment}Working pointer, which may be used freely for any
purposes. The application program can change this field directly.
\end{comment}

\noindent
\verb|glp_arc *t_next;|

\begin{comment}Pointer to another arc, which has the same tail endpoint
as this one. \verb|NULL| in this field indicates the end of the list of
outgoing arcs.
\end{comment}

\noindent
\verb|glp_arc *h_next;|

\begin{comment}Pointer to another arc, which has the same head endpoint
as this one. \verb|NULL| in this field indicates the end of the list of
incoming arcs.
\end{comment}

%%%%%%%%%%%%%%%%%%%%%%%%%%%%%%%%%%%%%%%%%%%%%%%%%%%%%%%%%%%%%%%%%%%%%%%%

\newpage

\section{Graph creating and modifying routines}

\subsection{glp\_create\_graph --- create graph}

\synopsis

\begin{verbatim}
   glp_graph *glp_create_graph(int v_size, int a_size);
\end{verbatim}

\description

The routine \verb|glp_create_graph| creates a new graph, which
initially is empty, i.e. has no vertices and arcs.

The parameter \verb|v_size| specifies the size of vertex data blocks,
in bytes, $0\leq v\_size\leq 256$.

The parameter \verb|a_size| specifies the size of arc data blocks, in
bytes, $0\leq a\_size\leq 256$.

\returns

The routine returns a pointer to the graph object created.

\subsection{glp\_set\_graph\_name --- assign (change) graph name}

\synopsis

\begin{verbatim}
   void glp_set_graph_name(glp_graph *G, const char *name);
\end{verbatim}

\description

The routine \verb|glp_set_graph_name| assigns a symbolic name specified
by the character string \verb|name| (1 to 255 chars) to the graph.

If the parameter \verb|name| is \verb|NULL| or an empty string, the
routine erases the existing symbolic name of the graph.

\subsection{glp\_add\_vertices --- add new vertices to graph}

\synopsis

\begin{verbatim}
   int glp_add_vertices(glp_graph *G, int nadd);
\end{verbatim}

\description

The routine \verb|glp_add_vertices| adds \verb|nadd| vertices to the
specified graph. New vertices are always added to the end of the vertex
list, so ordinal numbers of existing vertices remain unchanged. Note
that this operation may change the field \verb|v| in the struct
\verb|glp_graph| (pointer to the vertex array) due to reallocation.

Being added each new vertex is isolated, i.e. has no incident arcs.

If the size of vertex data blocks specified on creating the graph is
non-zero, the routine also allocates a memory block of that size for
each new vertex added, fills it by binary zeros, and stores a pointer
to it in the field \verb|data| of the struct \verb|glp_vertex|.
Otherwise, if the block size is zero, the field \verb|data| is set to
\verb|NULL|.

\returns

The routine \verb|glp_add_vertices| returns the ordinal number of the
first new vertex added to the graph.

\subsection{glp\_set\_vertex\_name --- assign (change) vertex name}

\synopsis

\begin{verbatim}
   void glp_set_vertex_name(glp_graph *G, int i, const char *name);
\end{verbatim}

\description

The routine \verb|glp_set_vertex_name| assigns a given symbolic name
(1 up to 255 characters) to \verb|i|-th vertex of the specified graph.

If the parameter \verb|name| is \verb|NULL| or empty string, the
routine erases an existing name of \verb|i|-th vertex.

\subsection{glp\_add\_arc --- add new arc to graph}

\synopsis

\begin{verbatim}
   glp_arc *glp_add_arc(glp_graph *G, int i, int j);
\end{verbatim}

\description

The routine \verb|glp_add_arc| adds one new arc to the specified graph.

The parameters \verb|i| and \verb|j| specify the ordinal numbers of,
resp., tail and head endpoints (vertices) of the arc. Note that
self-loops and multiple arcs are allowed.

If the size of arc data blocks specified on creating the graph is
non-zero, the routine also allocates a memory block of that size, fills
it by binary zeros, and stores a pointer to it in the field \verb|data|
of the struct \verb|glp_arc|. Otherwise, if the block size is zero, the
field \verb|data| is set to \verb|NULL|.

\subsection{glp\_del\_vertices --- delete vertices from graph}

\synopsis

\begin{verbatim}
   void glp_del_vertices(glp_graph *G, int ndel, const int num[]);
\end{verbatim}

\description

The routine \verb|glp_del_vertices| deletes vertices along with all
incident arcs from the specified graph. Ordinal numbers of vertices to
be deleted should be placed in locations \verb|num[1]|, \dots,
\verb|num[ndel]|, \verb|ndel| $>0$.

Note that deleting vertices involves changing ordinal numbers of other
vertices remaining in the graph. New ordinal numbers of the remaining
vertices are assigned under the assumption that the original order of
vertices is not changed.

\newpage

\subsection{glp\_del\_arc --- delete arc from graph}

\synopsis

\begin{verbatim}
   void glp_del_arc(glp_graph *G, glp_arc *a);
\end{verbatim}

\description

The routine \verb|glp_del_arc| deletes an arc from the specified graph.
The arc to be deleted must exist.

\subsection{glp\_erase\_graph --- erase graph content}

\synopsis

\begin{verbatim}
   void glp_erase_graph(glp_graph *G, int v_size, int a_size);
\end{verbatim}

\description

The routine \verb|glp_erase_graph| erases the content of the specified
graph. The effect of this operation is the same as if the graph would
be deleted with the routine \verb|glp_delete_graph| and then created
anew with the routine \verb|glp_create_graph|, with exception that the
pointer to the graph remains valid.

The parameters \verb|v_size| and \verb|a_size| have the same meaning as
for \verb|glp_create_graph|.

\subsection{glp\_delete\_graph --- delete graph}

\synopsis

\begin{verbatim}
   void glp_delete_graph(glp_graph *G);
\end{verbatim}

\description

The routine \verb|glp_delete_graph| deletes the specified graph and
frees all the memory allocated to this program object.

%%%%%%%%%%%%%%%%%%%%%%%%%%%%%%%%%%%%%%%%%%%%%%%%%%%%%%%%%%%%%%%%%%%%%%%%

\newpage

\section{Graph searching routines}

\subsection{glp\_create\_v\_index --- create vertex name index}

\synopsis

\begin{verbatim}
   void glp_create_v_index(glp_graph *G);
\end{verbatim}

\description

The routine \verb|glp_create_v_index| creates the name index for the
specified graph. The name index is an auxiliary data structure, which
is intended to quickly (i.e. for logarithmic time) find vertices by
their names.

This routine can be called at any time. If the name index already
exists, the routine does nothing.

\subsection{glp\_find\_vertex --- find vertex by its name}

\synopsis

\begin{verbatim}
   int glp_find_vertex(glp_graph *G, const char *name);
\end{verbatim}

\returns

The routine \verb|glp_find_vertex| returns the ordinal number of
a vertex, which is assigned (by the routine \verb|glp_set_vertex_name|)
the specified symbolic \verb|name|. If no such vertex exists, the
routine returns 0.

\subsection{glp\_delete\_v\_index --- delete vertex name index}

\synopsis

\begin{verbatim}
   void glp_delete_v_index(glp_graph *G);
\end{verbatim}

\description

The routine \verb|glp_delete_v_index| deletes the name index previously
created by the routine \verb|glp_create_v_index| and frees the memory
allocated to this auxiliary data structure.

This routine can be called at any time. If the name index does not
exist, the routine does nothing.

%%%%%%%%%%%%%%%%%%%%%%%%%%%%%%%%%%%%%%%%%%%%%%%%%%%%%%%%%%%%%%%%%%%%%%%%

\newpage

\section{Graph reading/writing routines}

\subsection{glp\_read\_graph --- read graph from plain text file}

\synopsis

\begin{verbatim}
   int glp_read_graph(glp_graph *G, const char *fname);
\end{verbatim}

\description

The routine \verb|glp_read_graph| reads a graph from a plain text file,
whose name is specified by the parameter \verb|fname|. Note that before
reading data the current content of the graph object is completely
erased with the routine \verb|glp_erase_graph|.

For the file format see description of the routine
\verb|glp_write_graph|.

\returns

If the operation was successful, the routine returns zero. Otherwise
it prints an error message and returns non-zero.

\subsection{glp\_write\_graph --- write graph to plain text file}

\synopsis

\begin{verbatim}
   int glp_write_graph(glp_graph *G, const char *fname);
\end{verbatim}

\description

The routine \verb|glp_write_graph| writes the graph to a plain text
file, whose name is specified by the parameter \verb|fname|.

\returns

If the operation was successful, the routine returns zero. Otherwise
it prints an error message and returns non-zero.

\para{File format}

The file created by the routine \verb|glp_write_graph| is a plain text
file, which contains the following information:

\begin{verbatim}
   nv na
   i[1] j[1]
   i[2] j[2]
   . . .
   i[na] j[na]
\end{verbatim}

\noindent
where: \verb|nv| is the number of vertices (nodes); \verb|na| is the
number of arcs; \verb|i[k]|, $k=1,\dots,na$, is the index of tail
vertex of arc $k$; \verb|j[k]|, $k=1,\dots,na$, is the index of head
vertex of arc $k$.

\newpage

\subsection{glp\_read\_ccdata --- read graph from text file in DIMACS
clique/coloring\\format}

\synopsis

\begin{verbatim}
   int glp_read_ccdata(glp_graph *G, int v_wgt, const char *fname);
\end{verbatim}

\description

The routine {\tt glp\_read\_ccdata} reads a graph from a text file in
DIMACS clique/coloring format. (Though this format is originally
designed to represent data for the minimal vertex coloring and maximal
clique problems, it may be used to represent general undirected and
directed graphs, because the routine allows reading self-loops and
multiple edges/arcs keeping the order of vertices specified for each
edge/arc of the graph.)

The parameter {\tt G} specifies the graph object to be read in. Note
that before reading data the current content of the graph object is
completely erased with the routine {\tt glp\_erase\_graph}.

The parameter {\tt v\_wgt} specifies an offset of the field of type
{\tt double} in the vertex data block, to which the routine stores the
vertex weight. If {\tt v\_wgt} $<0$, the vertex weights are not stored.

The character string {\tt fname} specifies the name of a text file to
be read in. (If the file name ends with the suffix `\verb|.gz|', the
file is assumed to be compressed, in which case the routine
decompresses it ``on the fly''.)

\returns

If the operation was successful, the routine returns zero. Otherwise,
it prints an error message and returns non-zero.

\para{DIMACS clique/coloring format\footnote{This material is
based on the paper ``Clique and Coloring Problems Graph Format'', which
is publically available at \url{http://dimacs.rutgers.edu/Challenges}.}}

The DIMACS input file is a plain ASCII text file. It contains
{\it lines} of several types described below. A line is terminated with
an end-of-line character. Fields in each line are separated by at least
one blank space. Each line begins with a one-character designator to
identify the line type.

Note that DIMACS requires all numerical quantities to be integers in
the range $[-2^{31},2^{31}-1]$ while GLPK allows the quantities to be
floating-point numbers.

\para{Comment lines.} Comment lines give human-readable information
about the file and are ignored by programs. Comment lines can appear
anywhere in the file. Each comment line begins with a lower-case
character \verb|c|.

\begin{verbatim}
   c This is a comment line
\end{verbatim}

\para{Problem line.} There is one problem line per data file.
The problem line must appear before any node or edge descriptor lines.
It has the following format:

\begin{verbatim}
   p edge NODES EDGES
\end{verbatim}

\noindent
The lower-case letter \verb|p| signifies that this is a problem line.
The four-character problem designator \verb|edge| identifies the file
as containing data for the minimal vertex coloring or maximal clique
problem. The \verb|NODES| field contains an integer value specifying
the number of vertices in the graph. The \verb|EDGES| field contains an
integer value specifying the number of edges (arcs) in the graph.

\para{Vertex descriptors.} These lines give the weight assigned to
a vertex of the graph. There is one vertex descriptor line for each
vertex, with the following format. Vertices without a descriptor take
on a default value of 1.

\begin{verbatim}
   n ID VALUE
\end{verbatim}

\noindent
The lower-case character \verb|n| signifies that this is a vertex
descriptor line. The \verb|ID| field gives a vertex identification
number, an integer between 1 and $n$, where $n$ is the number of
vertices in the graph. The \verb|VALUE| field gives a vertex weight,
which can either positive or negative (or zero).

\para{Edge descriptors.} There is one edge descriptor line for each
edge (arc) of the graph, each with the following format:

\begin{verbatim}
   e I J
\end{verbatim}

\noindent
The lower-case character \verb|e| signifies that this is an edge
descriptor line. For an edge (arc) $(i,j)$ the fields \verb|I| and
\verb|J| specify its endpoints.

\para{Example.} The following example of an undirected graph:

\bigskip

\noindent\hfil
\xymatrix %@C=32pt
{&{v_1}\ar@{-}[ldd]\ar@{-}[dd]\ar@{-}[rd]\ar@{-}[rr]&&{v_2}\ar@{-}[ld]
\ar@{-}[dd]\ar@{-}[rdd]&\\
&&{v_7}\ar@{-}[ld]\ar@{-}[rd]&&\\
{v_6}\ar@{-}[r]\ar@{-}[rdd]&{v_{10}}\ar@{-}[rr]\ar@{-}[rd]\ar@{-}[dd]&&
{v_8}\ar@{-}[ld]\ar@{-}[dd]\ar@{-}[r]&{v_3}\ar@{-}[ldd]\\
&&{v_9}\ar@{-}[ld]\ar@{-}[rd]&&\\
&{v_5}\ar@{-}[rr]&&{v_4}&\\
}

\bigskip

\noindent
might be coded in DIMACS clique/coloring format as follows:

\begin{footnotesize}
\begin{verbatim}
c sample.col
c
c This is an example of the vertex coloring problem data
c in DIMACS format.
c
p edge 10 21
c
e 1 2
e 1 6
e 1 7
e 1 10
e 2 3
e 2 7
e 2 8
e 3 4
e 3 8
e 4 5
e 4 8
e 4 9
e 5 6
e 5 9
e 5 10
e 6 10
e 7 8
e 7 10
e 8 9
e 8 10
e 9 10
c
c eof
\end{verbatim}
\end{footnotesize}

\subsection{glp\_write\_ccdata --- write graph to text file in DIMACS
clique/coloring\\format}

\synopsis

\begin{verbatim}
   int glp_write_ccdata(glp_graph *G, int v_wgt, const char *fname);
\end{verbatim}

\description

The routine {\tt glp\_write\_ccdata} writes the graph object specified
by the parameter {\tt G} to a text file in DIMACS clique/coloring
format. (Though this format is originally designed to represent data
for the minimal vertex coloring and maximal clique problems, it may be
used to represent general undirected and directed graphs, because the
routine allows writing self-loops and multiple edges/arcs keeping the
order of vertices specified for each edge/arc of the graph.)

The parameter {\tt v\_wgt} specifies an offset of the field of type
{\tt double} in the vertex data block, which contains the vertex
weight. If {\tt v\_wgt} $<0$, it is assumed that the weight of each
vertex is 1.

The character string {\tt fname} specifies a name of the text file to
be written out. (If the file name ends with suffix `\verb|.gz|', the
file is assumed to be compressed, in which case the routine performs
automatic compression on writing it.)

\returns

If the operation was successful, the routine returns zero. Otherwise,
it prints an error message and returns non-zero.

%%%%%%%%%%%%%%%%%%%%%%%%%%%%%%%%%%%%%%%%%%%%%%%%%%%%%%%%%%%%%%%%%%%%%%%%

\newpage

\section{Graph analysis routines}

\subsection{glp\_weak\_comp --- find all weakly connected components of
graph}

\synopsis

\begin{verbatim}
   int glp_weak_comp(glp_graph *G, int v_num);
\end{verbatim}

\description

The routine \verb|glp_weak_comp| finds all weakly connected components
of the specified graph.

The parameter \verb|v_num| specifies an offset of the field of type
\verb|int| in the vertex data block, to which the routine stores the
number of a weakly connected component containing that vertex. If
\verb|v_num| $<0$, no component numbers are stored.

The components are numbered in arbitrary order from 1 to \verb|nc|,
where \verb|nc| is the total number of components found,
$0\leq$ \verb|nc| $\leq|V|$.

\returns

The routine returns \verb|nc|, the total number of components found.

\subsection{glp\_strong\_comp --- find all strongly connected
components of graph}

\synopsis

\begin{verbatim}
   int glp_strong_comp(glp_graph *G, int v_num);
\end{verbatim}

\description

The routine \verb|glp_strong_comp| finds all strongly connected
components of the specified graph.

The parameter \verb|v_num| specifies an offset of the field of type
\verb|int| in the vertex data block, to which the routine stores the
number of a strongly connected component containing that vertex. If
\verb|v_num| $<0$, no component numbers are stored.

The components are numbered in arbitrary order from 1 to \verb|nc|,
where \verb|nc| is the total number of components found,
$0\leq$ \verb|nc| $\leq|V|$. However, the component numbering has the
property that for every arc $(i\rightarrow j)$ in the graph the
condition $num(i)\geq num(j)$ holds.

\returns

The routine returns \verb|nc|, the total number of components found.

\para{References}

I.~S.~Duff, J.~K.~Reid, Algorithm 529: Permutations to block triangular
form, ACM Trans. on Math. Softw. 4 (1978), 189-92.

\newpage

\para{Example}

The following program reads a graph from a plain text file
`\verb|graph.txt|' and finds all its strongly connected components.

\begin{footnotesize}
\begin{verbatim}
#include <stddef.h>
#include <stdio.h>
#include <stdlib.h>
#include <glpk.h>

typedef struct { int num; } v_data;

#define vertex(v) ((v_data *)((v)->data))

int main(void)
{     glp_graph *G;
      int i, nc;
      G = glp_create_graph(sizeof(v_data), 0);
      glp_read_graph(G, "graph.txt");
      nc = glp_strong_comp(G, offsetof(v_data, num));
      printf("nc = %d\n", nc);
      for (i = 1; i <= G->nv; i++)
         printf("num[%d] = %d\n", i, vertex(G->v[i])->num);
      glp_delete_graph(G);
      return 0;
}
\end{verbatim}
\end{footnotesize}

\noindent
If the file `\verb|graph.txt|' contains the following graph:

\medskip

\noindent\hfil
\xymatrix
{1\ar[r]&2\ar[r]&3\ar[r]\ar[dd]&4\ar[dd]\\
5\ar[u]&6\ar[l]\\
7\ar[u]&&8\ar[lu]\ar[ll]\ar[r]&9\ar[r]&10\ar[r]\ar[d]&11\ar[d]\\
12\ar[u]\ar[rru]\ar@/_/[rr]&&13\ar[ll]\ar[u]\ar[rr]&&14\ar[lu]&15\ar[l]
\\
}

\medskip\medskip

\noindent
the program output may look like follows:

\begin{footnotesize}
\begin{verbatim}
Reading graph from `graph.txt'...
Graph has 15 vertices and 30 arcs
31 lines were read
nc = 4
num[1] = 3
num[2] = 3
num[3] = 3
num[4] = 2
num[5] = 3
num[6] = 3
num[7] = 3
num[8] = 3
num[9] = 1
num[10] = 1
num[11] = 1
num[12] = 4
num[13] = 4
num[14] = 1
num[15] = 1
\end{verbatim}
\end{footnotesize}

\subsection{glp\_top\_sort --- topological sorting of acyclic digraph}

\synopsis

\begin{verbatim}
   int glp_top_sort(glp_graph *G, int v_num);
\end{verbatim}

\description

The routine \verb|glp_top_sort| performs topological sorting of
vertices of the specified acyclic digraph.

The parameter \verb|v_num| specifies an offset of the field of type
\verb|int| in the vertex data block, to which the routine stores the
vertex number assigned. If \verb|v_num| $<0$, vertex numbers are not
stored.

The vertices are numbered from 1 to $n$, where $n$ is the total number
of vertices in the graph. The vertex numbering has the property that
for every arc $(i\rightarrow j)$ in the graph the condition
$num(i)<num(j)$ holds. Special case $num(i)=0$ means that vertex $i$ is
not assigned a number, because the graph is {\it not} acyclic.

\returns

If the graph is acyclic and therefore all the vertices have been
assigned numbers, the routine \verb|glp_top_sort| returns zero.
Otherwise, if the graph is not acyclic, the routine returns the number
of vertices which have not been numbered, i.e. for which $num(i)=0$.

\para{Example}

The following program reads a digraph from a plain text file
`\verb|graph.txt|' and performs topological sorting of its vertices.

\begin{footnotesize}
\begin{verbatim}
#include <stddef.h>
#include <stdio.h>
#include <stdlib.h>
#include <glpk.h>

typedef struct { int num; } v_data;

#define vertex(v) ((v_data *)((v)->data))

int main(void)
{     glp_graph *G;
      int i, cnt;
      G = glp_create_graph(sizeof(v_data), 0);
      glp_read_graph(G, "graph.txt");
      cnt = glp_top_sort(G, offsetof(v_data, num));
      printf("cnt = %d\n", cnt);
      for (i = 1; i <= G->nv; i++)
         printf("num[%d] = %d\n", i, vertex(G->v[i])->num);
      glp_delete_graph(G);
      return 0;
}
\end{verbatim}
\end{footnotesize}

\noindent
If the file `\verb|graph.txt|' contains the following graph:

\medskip

\noindent\hfil
\xymatrix @=20pt
{
1\ar[rrr]&&&2\ar[r]\ar[rddd]&3\ar[rd]&&&&\\
&&&4\ar[ru]&&5\ar[r]&6\ar[rd]\ar[dd]&&\\
7\ar[r]&8\ar[r]&9\ar[ruu]\ar[ru]\ar[r]\ar[rd]&10\ar[rr]\ar[rru]&&
11\ar[ru]&&12\ar[r]&13\\
&&&14\ar[r]&15\ar[rrru]\ar[rr]&&16\ar[rru]\ar[rr]&&17\\
}

\medskip\medskip

\noindent
the program output may look like follows:

\begin{footnotesize}
\begin{verbatim}
Reading graph from `graph.txt'...
Graph has 17 vertices and 23 arcs
24 lines were read
cnt = 0
num[1] = 8
num[2] = 9
num[3] = 10
num[4] = 4
num[5] = 11
num[6] = 12
num[7] = 1
num[8] = 2
num[9] = 3
num[10] = 5
num[11] = 6
num[12] = 14
num[13] = 16
num[14] = 7
num[15] = 13
num[16] = 15
num[17] = 17
\end{verbatim}
\end{footnotesize}

\noindent
The output corresponds to the following vertex numbering:

\medskip

\noindent\hfil
\xymatrix @=20pt
{
8\ar[rrr]&&&9\ar[r]\ar[rddd]&10\ar[rd]&&&&\\
&&&4\ar[ru]&&11\ar[r]&12\ar[rd]\ar[dd]&&\\
1\ar[r]&2\ar[r]&3\ar[ruu]\ar[ru]\ar[r]\ar[rd]&5\ar[rr]\ar[rru]&&
6\ar[ru]&&14\ar[r]&16\\
&&&7\ar[r]&13\ar[rrru]\ar[rr]&&15\ar[rru]\ar[rr]&&17\\
}

%%%%%%%%%%%%%%%%%%%%%%%%%%%%%%%%%%%%%%%%%%%%%%%%%%%%%%%%%%%%%%%%%%%%%%%%

\chapter{Network optimization API routines}

\section{Minimum cost flow problem}

\subsection{Background}

The {\it minimum cost flow problem} (MCFP) is stated as follows. Let
there be given a directed graph (flow network) $G=(V,A)$, where $V$ is
a set of vertices (nodes), and $A\subseteq V\times V$ is a set of arcs.
Let for each node $i\in V$ there be given a quantity $b_i$ having the
following meaning:

if $b_i>0$, then $|b_i|$ is a {\it supply} at node $i$, which shows
how many flow units are {\it generated} at node $i$ (or, equivalently,
entering the network through node $i$ from outside);

if $b_i<0$, then $|b_i|$ is a {\it demand} at node $i$, which shows how
many flow units are {\it lost} at node $i$ (or, equivalently, leaving
the network through node $i$ to outside);

if $b_i=0$, then $i$ is a {\it transshipment} node, at which the flow
is conserved, i.e. neither generated nor lost.

Let also for each arc $a=(i,j)\in A$ there be given the following three
quantities:

$l_{ij}$, a (non-negative) lower bound to the flow through arc $(i,j)$;

$u_{ij}$, an upper bound to the flow through arc $(i,j)$, which is the
{\it arc capacity};

$c_{ij}$, a per-unit cost of the flow through arc $(i,j)$.

The problem is to find flows $x_{ij}$ through every arc of the network,
which satisfy the specified bounds and the conservation constraints at
all nodes, and minimize the total flow cost. Here the conservation
constraint at a node means that the total flow entering this node
through its incoming arcs plus the supply at this node must be equal to
the total flow leaving this node through its outgoing arcs plus the
demand at this node.

An example of the minimum cost flow problem is shown on Fig.~1.

\newpage

\noindent\hfil
\xymatrix @C=48pt
{_{20}\ar@{~>}[d]&
v_2\ar[r]|{_{0,10,\$2}}\ar[dd]|{_{0,9,\$3}}&
v_3\ar[dd]|{_{2,12,\$1}}\ar[r]|{_{0,18,\$0}}&
v_8\ar[rd]|{_{0,20,\$9}}&\\
v_1\ar[ru]|{_{0,14,\$0}}\ar[rd]|{_{0,23,\$0}}&&&
v_6\ar[d]|{_{0,7,\$0}}\ar[u]|{_{4,8,\$0}}&
v_9\ar@{~>}[d]\\
&v_4\ar[r]|{_{0,26,\$0}}&
v_5\ar[luu]|{_{0,11,\$1}}\ar[ru]|{_{0,25,\$5}}\ar[r]|{_{0,4,\$7}}&
v_7\ar[ru]|{_{0,15,\$3}}&_{20}\\
}

\noindent\hfil
\begin{tabular}{ccc}
\xymatrix @C=48pt{v_i\ar[r]|{\ l,u,\$c\ }&v_j\\}&
\xymatrix{\hbox{\footnotesize supply}\ar@{~>}[r]&v_i\\}&
\xymatrix{v_i\ar@{~>}[r]&\hbox{\footnotesize demand}\\}\\
\end{tabular}

\noindent\hfil
Fig.~1. An example of the minimum cost flow problem.

\medskip

The minimum cost flow problem can be naturally formulated as the
following LP problem:

\noindent
\hspace{1in}minimize
$$z=\sum_{(i,j)\in A}c_{ij}x_{ij}\eqno(1)$$
\hspace{1in}subject to
$$\sum_{(i,j)\in A}x_{ij}-\sum_{(j,i)\in A}x_{ji}=b_i\ \ \ \hbox
{for all}\ i\in V\eqno(2)$$
$$l_{ij}\leq x_{ij}\leq u_{ij}\ \ \ \hbox{for all}\ (i,j)\in A
\eqno(3)$$

\subsection{glp\_read\_mincost --- read minimum cost flow problem data
in DIMACS\\format}

\synopsis

\begin{verbatim}
   int glp_read_mincost(glp_graph *G, int v_rhs, int a_low, int a_cap,
                        int a_cost, const char *fname);
\end{verbatim}

\description

The routine \verb|glp_read_mincost| reads the minimum cost flow problem
data from a text file in DIMACS format.

The parameter \verb|G| specifies the graph object, to which the problem
data have to be stored. Note that before reading data the current
content of the graph object is completely erased with the routine
\verb|glp_erase_graph|.

The parameter \verb|v_rhs| specifies an offset of the field of type
\verb|double| in the vertex data block, to which the routine stores
$b_i$, the supply/demand value. If \verb|v_rhs| $<0$, the value is not
stored.

The parameter \verb|a_low| specifies an offset of the field of type
\verb|double| in the arc data block, to which the routine stores
$l_{ij}$, the lower bound to the arc flow. If \verb|a_low| $<0$, the
lower bound is not stored.

The parameter \verb|a_cap| specifies an offset of the field of type
\verb|double| in the arc data block, to which the routine stores
$u_{ij}$, the upper bound to the arc flow (the arc capacity). If
\verb|a_cap| $<0$, the upper bound is not stored.

The parameter \verb|a_cost| specifies an offset of the field of type
\verb|double| in the arc data block, to which the routine stores
$c_{ij}$, the per-unit cost of the arc flow. If \verb|a_cost| $<0$, the
cost is not stored.

The character string \verb|fname| specifies the name of a text file to
be read in. (If the file name name ends with the suffix `\verb|.gz|',
the file is assumed to be compressed, in which case the routine
decompresses it ``on the fly''.)

\returns

If the operation was successful, the routine returns zero. Otherwise,
it prints an error message and returns non-zero.

\para{Example}

\begin{footnotesize}
\begin{verbatim}
typedef struct
{     /* vertex data block */
      ...
      double rhs;
      ...
} v_data;

typedef struct
{     /* arc data block */
      ...
      double low, cap, cost;
      ...
} a_data;

int main(void)
{     glp_graph *G;
      int ret;
      G = glp_create_graph(sizeof(v_data), sizeof(a_data));
      ret = glp_read_mincost(G, offsetof(v_data, rhs),
         offsetof(a_data, low), offsetof(a_data, cap),
         offsetof(a_data, cost), "sample.min");
      if (ret != 0) goto ...
      ...
}
\end{verbatim}
\end{footnotesize}

\para{DIMACS minimum cost flow problem format\footnote{This
material is based on the paper ``The First DIMACS International
Algorithm Implementation Challenge: Problem Definitions and
Specifications'', which is publically available at
\url{http://dimacs.rutgers.edu/Challenges}.}}
\label{subsecmincost}

The DIMACS input file is a plain ASCII text file. It contains
{\it lines} of several types described below. A line is terminated with
an end-of-line character. Fields in each line are separated by at least
one blank space. Each line begins with a one-character designator to
identify the line type.

Note that DIMACS requires all numerical quantities to be integers in
the range $[-2^{31},\ 2^{31}-1]$ while GLPK allows the quantities to be
floating-point numbers.

\para{Comment lines.} Comment lines give human-readable information
about the file and are ignored by programs. Comment lines can appear
anywhere in the file. Each comment line begins with a lower-case
character \verb|c|.

\begin{verbatim}
   c This is a comment line
\end{verbatim}

\newpage

\para{Problem line.} There is one problem line per data file. The
problem line must appear before any node or arc descriptor lines. It
has the following format:

\begin{verbatim}
   p min NODES ARCS
\end{verbatim}

\noindent
The lower-case character \verb|p| signifies that this is a problem line.
The three-character problem designator \verb|min| identifies the file as
containing specification information for the minimum cost flow problem.
The \verb|NODES| field contains an integer value specifying the number
of nodes in the network. The \verb|ARCS| field contains an integer value
specifying the number of arcs in the network.

\para{Node descriptors.} All node descriptor lines must appear before
all arc descriptor lines. The node descriptor lines describe supply and
demand nodes, but not transshipment nodes. That is, only nodes with
non-zero node supply/demand values appear. There is one node descriptor
line for each such node, with the following format:

\begin{verbatim}
   n ID FLOW
\end{verbatim}

\noindent
The lower-case character \verb|n| signifies that this is a node
descriptor line. The \verb|ID| field gives a node identification
number, an integer between 1 and \verb|NODES|. The \verb|FLOW| field
gives the amount of supply (if positive) or demand (if negative) at
node \verb|ID|.

\para{Arc descriptors.} There is one arc descriptor line for each arc
in the network. Arc descriptor lines are of the following format:

\begin{verbatim}
   a SRC DST LOW CAP COST
\end{verbatim}

\noindent
The lower-case character \verb|a| signifies that this is an arc
descriptor line. For a directed arc $(i,j)$ the \verb|SRC| field gives
the identification number $i$ for the tail endpoint, and the \verb|DST|
field gives the identification number $j$ for the head endpoint.
Identification numbers are integers between 1 and \verb|NODES|. The
\verb|LOW| field specifies the minimum amount of flow that can be sent
along arc $(i,j)$, and the \verb|CAP| field gives the maximum amount of
flow that can be sent along arc $(i,j)$ in a feasible flow. The
\verb|COST| field contains the per-unit cost of flow sent along arc
$(i,j)$.

\para{Example.} Below here is an example of the data file in DIMACS
format corresponding to the minimum cost flow problem shown on Fig~1.

\begin{footnotesize}
\begin{verbatim}
c sample.min
c
c This is an example of the minimum cost flow problem data
c in DIMACS format.
c
p min 9 14
c
n 1 20
n 9 -20
c
a 1 2 0 14 0
a 1 4 0 23 0
a 2 3 0 10 2
a 2 4 0  9 3
a 3 5 2 12 1
a 3 8 0 18 0
a 4 5 0 26 0
a 5 2 0 11 1
a 5 6 0 25 5
a 5 7 0  4 7
a 6 7 0  7 0
a 6 8 4  8 0
a 7 9 0 15 3
a 8 9 0 20 9
c
c eof
\end{verbatim}
\end{footnotesize}

\subsection{glp\_write\_mincost --- write minimum cost flow problem
data in DIMACS\\format}

\synopsis

\begin{verbatim}
   int glp_write_mincost(glp_graph *G, int v_rhs, int a_low, int a_cap,
                         int a_cost, const char *fname);
\end{verbatim}

\description

The routine \verb|glp_write_mincost| writes the minimum cost flow
problem data to a text file in DIMACS format.

The parameter \verb|G| is the graph (network) program object, which
specifies the minimum cost flow problem instance.

The parameter \verb|v_rhs| specifies an offset of the field of type
\verb|double| in the vertex data block, which contains $b_i$, the
supply/demand value. If \verb|v_rhs| $<0$, it is assumed that $b_i=0$
for all nodes.

The parameter \verb|a_low| specifies an offset of the field of type
\verb|double| in the arc data block, which contains $l_{ij}$, the lower
bound to the arc flow. If \verb|a_low| $<0$, it is assumed that
$l_{ij}=0$ for all arcs.

The parameter \verb|a_cap| specifies an offset of the field of type
\verb|double| in the arc data block, which contains $u_{ij}$, the upper
bound to the arc flow (the arc capacity). If the upper bound is
specified as \verb|DBL_MAX|, it is assumed that $u_{ij}=\infty$, i.e.
the arc is uncapacitated. If \verb|a_cap| $<0$, it is assumed that
$u_{ij}=1$ for all arcs.

The parameter \verb|a_cost| specifies an offset of the field of type
\verb|double| in the arc data block, which contains $c_{ij}$, the
per-unit cost of the arc flow. If \verb|a_cost| $<0$, it is assumed
that $c_{ij}=0$ for all arcs.

The character string \verb|fname| specifies a name of the text file to
be written out. (If the file name ends with suffix `\verb|.gz|', the
file is assumed to be compressed, in which case the routine performs
automatic compression on writing it.)

\returns

If the operation was successful, the routine returns zero. Otherwise,
it prints an error message and returns non-zero.

\newpage

\subsection{glp\_mincost\_lp --- convert minimum cost flow problem
to LP}

\synopsis

\begin{verbatim}
   void glp_mincost_lp(glp_prob *P, glp_graph *G, int names, int v_rhs,
                       int a_low, int a_cap, int a_cost);
\end{verbatim}

\description

The routine \verb|glp_mincost_lp| builds LP problem (1)---(3), which
corresponds to the specified minimum cost flow problem.

The parameter \verb|P| is the resultant LP problem object to be built.
Note that on entry its current content is erased with the routine
\verb|glp_erase_prob|.

The parameter \verb|G| is the graph (network) program object, which
specifies the minimum cost flow problem instance.

The parameter \verb|names| is a flag. If it is \verb|GLP_ON|, the
routine uses symbolic names of the graph object components to assign
symbolic names to the LP problem object components. If the flag is
\verb|GLP_OFF|, no symbolic names are assigned.

The parameter \verb|v_rhs| specifies an offset of the field of type
\verb|double| in the vertex data block, which contains $b_i$, the
supply/demand value. If \verb|v_rhs| $<0$, it is assumed that $b_i=0$
for all nodes.

The parameter \verb|a_low| specifies an offset of the field of type
\verb|double| in the arc data block, which contains $l_{ij}$, the lower
bound to the arc flow. If \verb|a_low| $<0$, it is assumed that
$l_{ij}=0$ for all arcs.

The parameter \verb|a_cap| specifies an offset of the field of type
\verb|double| in the arc data block, which contains $u_{ij}$, the upper
bound to the arc flow (the arc capacity). If the upper bound is
specified as \verb|DBL_MAX|, it is assumed that $u_{ij}=\infty$, i.e.
the arc is uncapacitated. If \verb|a_cap| $<0$, it is assumed that
$u_{ij}=1$ for all arcs.

The parameter \verb|a_cost| specifies an offset of the field of type
\verb|double| in the arc data block, which contains $c_{ij}$, the
per-unit cost of the arc flow. If \verb|a_cost| $<0$, it is assumed that
$c_{ij}=0$ for all arcs.

\para{Example}

The example program below reads the minimum cost problem instance in
DIMACS format from file `\verb|sample.min|', converts the instance to
LP, and then writes the resultant LP in CPLEX format to file
`\verb|mincost.lp|'.

\begin{footnotesize}
\begin{verbatim}
#include <stddef.h>
#include <glpk.h>

typedef struct { double rhs; } v_data;
typedef struct { double low, cap, cost; } a_data;

int main(void)
{     glp_graph *G;
      glp_prob *P;
      G = glp_create_graph(sizeof(v_data), sizeof(a_data));
      glp_read_mincost(G, offsetof(v_data, rhs),
         offsetof(a_data, low), offsetof(a_data, cap),
         offsetof(a_data, cost), "sample.min");
      P = glp_create_prob();
      glp_mincost_lp(P, G, GLP_ON, offsetof(v_data, rhs),
         offsetof(a_data, low), offsetof(a_data, cap),
         offsetof(a_data, cost));
      glp_delete_graph(G);
      glp_write_lp(P, NULL, "mincost.lp");
      glp_delete_prob(P);
      return 0;
}
\end{verbatim}
\end{footnotesize}

If `\verb|sample.min|' is the example data file from the subsection
describing \verb|glp_read_mincost|, file `\verb|mincost.lp|' may look
like follows:

\begin{footnotesize}
\begin{verbatim}
Minimize
 obj: + 3 x(2,4) + 2 x(2,3) + x(3,5) + 7 x(5,7) + 5 x(5,6)
 + x(5,2) + 3 x(7,9) + 9 x(8,9)

Subject To
 r_1: + x(1,2) + x(1,4) = 20
 r_2: - x(5,2) + x(2,3) + x(2,4) - x(1,2) = 0
 r_3: + x(3,5) + x(3,8) - x(2,3) = 0
 r_4: + x(4,5) - x(2,4) - x(1,4) = 0
 r_5: + x(5,2) + x(5,6) + x(5,7) - x(4,5) - x(3,5) = 0
 r_6: + x(6,7) + x(6,8) - x(5,6) = 0
 r_7: + x(7,9) - x(6,7) - x(5,7) = 0
 r_8: + x(8,9) - x(6,8) - x(3,8) = 0
 r_9: - x(8,9) - x(7,9) = -20

Bounds
 0 <= x(1,4) <= 23
 0 <= x(1,2) <= 14
 0 <= x(2,4) <= 9
 0 <= x(2,3) <= 10
 0 <= x(3,8) <= 18
 2 <= x(3,5) <= 12
 0 <= x(4,5) <= 26
 0 <= x(5,7) <= 4
 0 <= x(5,6) <= 25
 0 <= x(5,2) <= 11
 4 <= x(6,8) <= 8
 0 <= x(6,7) <= 7
 0 <= x(7,9) <= 15
 0 <= x(8,9) <= 20

End
\end{verbatim}
\end{footnotesize}

\newpage

\subsection{glp\_mincost\_okalg --- solve minimum cost flow problem
with out-of-kilter\\algorithm}

\synopsis

\begin{verbatim}
   int glp_mincost_okalg(glp_graph *G, int v_rhs, int a_low, int a_cap,
                         int a_cost, double *sol, int a_x, int v_pi);
\end{verbatim}

\description

The routine \verb|glp_mincost_okalg| finds optimal solution to the
minimum cost flow problem with the out-of-kilter
algorithm.\footnote{GLPK implementation of the out-of-kilter algorithm
is based on the following book: L.~R.~Ford,~Jr., and D.~R.~Fulkerson,
``Flows in Networks,'' The RAND Corp., Report R-375-PR (August 1962),
Chap. III ``Minimal Cost Flow Problems,'' pp.~113-26.} Note that this
routine requires all the problem data to be integer-valued.

The parameter \verb|G| is a graph (network) program object which
specifies the minimum cost flow problem instance to be solved.

The parameter \verb|v_rhs| specifies an offset of the field of type
\verb|double| in the vertex data block, which contains $b_i$, the
supply/demand value. This value must be integer in the range
[$-$\verb|INT_MAX|, $+$\verb|INT_MAX|]. If \verb|v_rhs| $<0$, it is
assumed that $b_i=0$ for all nodes.

The parameter \verb|a_low| specifies an offset of the field of type
\verb|double| in the arc data block, which contains $l_{ij}$, the lower
bound to the arc flow. This bound must be integer in the range
[$0$, \verb|INT_MAX|]. If \verb|a_low| $<0$, it is assumed that
$l_{ij}=0$ for all arcs.

The parameter \verb|a_cap| specifies an offset of the field of type
\verb|double| in the arc data block, which contains $u_{ij}$, the upper
bound to the arc flow (the arc capacity). This bound must be integer in
the range [$l_{ij}$, \verb|INT_MAX|]. If \verb|a_cap| $<0$, it is
assumed that $u_{ij}=1$ for all arcs.

The parameter \verb|a_cost| specifies an offset of the field of type
\verb|double| in the arc data block, which contains $c_{ij}$, the
per-unit cost of the arc flow. This value must be integer in the range
[$-$\verb|INT_MAX|, $+$\verb|INT_MAX|]. If \verb|a_cost| $<0$, it is
assumed that $c_{ij}=0$ for all arcs.

The parameter \verb|sol| specifies a location, to which the routine
stores the objective value (that is, the total cost) found. If
\verb|sol| is NULL, the objective value is not stored.

The parameter \verb|a_x| specifies an offset of the field of type
\verb|double| in the arc data block, to which the routine stores
$x_{ij}$, the arc flow found. If \verb|a_x| $<0$, the arc flow value is
not stored.

The parameter \verb|v_pi| specifies an offset of the field of type
\verb|double| in the vertex data block, to which the routine stores
$\pi_i$, the node potential, which is the Lagrange multiplier for the
corresponding flow conservation equality constraint (see (2) in
Subsection ``Background''). If necessary, the application program may
use the node potentials to compute $\lambda_{ij}$, reduced costs of the
arc flows $x_{ij}$, which are the Lagrange multipliers for the arc flow
bound constraints (see (3) ibid.), using the following formula:
$$\lambda_{ij}=c_{ij}-(\pi_i-\pi_j),$$
where $c_{ij}$ is the per-unit cost for arc $(i,j)$.

\newpage

Note that all solution components (the objective value, arc flows, and
node potentials) computed by the routine are always integer-valued.

\returns

\begin{retlist}
0 & Optimal solution found.\\

\verb|GLP_ENOPFS| & No (primal) feasible solution exists.\\

\verb|GLP_EDATA| & Unable to start the search, because some problem
data are either not integer-valued or out of range. This code is also
returned if the total supply, which is the sum of $b_i$ over all source
nodes (nodes with $b_i>0$), exceeds \verb|INT_MAX|.\\

\verb|GLP_ERANGE| & The search was prematurely terminated because of
integer overflow.\\

\verb|GLP_EFAIL| & An error has been detected in the program logic.
(If this code is returned for your problem instance, please report to
\verb|<bug-glpk@gnu.org>|.)\\
\end{retlist}

\para{Comments}

By design the out-of-kilter algorithm is applicable only to networks,
where $b_i=0$ for {\it all} nodes, i.e. actually this algorithm finds a
minimal cost {\it circulation}. Due to this requirement the routine
\verb|glp_mincost_okalg| converts the original network to a network
suitable for the out-of-kilter algorithm in the following
way:\footnote{The conversion is performed internally and does not change
the original network program object passed to the routine.}

1) it adds two auxiliary nodes $s$ and $t$;

2) for each original node $i$ with $b_i>0$ the routine adds auxiliary
supply arc $(s\rightarrow i)$, flow $x_{si}$ through which is costless
($c_{si}=0$) and fixed to $+b_i$ ($l_{si}=u_{si}=+b_i$);

3) for each original node $i$ with $b_i<0$ the routine adds auxiliary
demand arc $(i\rightarrow t)$, flow $x_{it}$ through which is costless
($c_{it}=0$) and fixed to $-b_i$ ($l_{it}=u_{it}=-b_i$);

4) finally, the routine adds auxiliary feedback arc $(t\rightarrow s)$,
flow $x_{ts}$ through which is also costless ($c_{ts}=0$) and fixed to
$F$ ($l_{ts}=u_{ts}=F$), where $\displaystyle F=\sum_{b_i>0}b_i$ is the
total supply.

\para{Example}

The example program below reads the minimum cost problem instance in
DIMACS format from file `\verb|sample.min|', solves it by using the
routine \verb|glp_mincost_okalg|, and writes the solution found on the
standard output.

\begin{footnotesize}
\begin{verbatim}
#include <stddef.h>
#include <stdio.h>
#include <stdlib.h>
#include <glpk.h>

typedef struct { double rhs, pi; } v_data;
typedef struct { double low, cap, cost, x; } a_data;

#define node(v) ((v_data *)((v)->data))
#define arc(a)  ((a_data *)((a)->data))

int main(void)
{     glp_graph *G;
      glp_vertex *v, *w;
      glp_arc *a;
      int i, ret;
      double sol;
      G = glp_create_graph(sizeof(v_data), sizeof(a_data));
      glp_read_mincost(G, offsetof(v_data, rhs),
         offsetof(a_data, low), offsetof(a_data, cap),
         offsetof(a_data, cost), "sample.min");
      ret = glp_mincost_okalg(G, offsetof(v_data, rhs),
         offsetof(a_data, low), offsetof(a_data, cap),
         offsetof(a_data, cost), &sol, offsetof(a_data, x),
         offsetof(v_data, pi));
      printf("ret = %d; sol = %5g\n", ret, sol);
      for (i = 1; i <= G->nv; i++)
      {  v = G->v[i];
         printf("node %d:    pi = %5g\n", i, node(v)->pi);
         for (a = v->out; a != NULL; a = a->t_next)
         {  w = a->head;
            printf("arc  %d->%d: x  = %5g; lambda = %5g\n",
               v->i, w->i, arc(a)->x,
               arc(a)->cost - (node(v)->pi - node(w)->pi));
         }
      }
      glp_delete_graph(G);
      return 0;
}
\end{verbatim}
\end{footnotesize}

If `\verb|sample.min|' is the example data file from the subsection
describing \verb|glp_read_mincost|, the output may look like follows:

\begin{footnotesize}
\begin{verbatim}
Reading min-cost flow problem data from `sample.min'...
Flow network has 9 nodes and 14 arcs
24 lines were read
ret = 0; sol =   213
node 1:    pi =   -12
arc  1->4: x  =    13; lambda =     0
arc  1->2: x  =     7; lambda =     0
node 2:    pi =   -12
arc  2->4: x  =     0; lambda =     3
arc  2->3: x  =     7; lambda =     0
node 3:    pi =   -14
arc  3->8: x  =     5; lambda =     0
arc  3->5: x  =     2; lambda =     3
node 4:    pi =   -12
arc  4->5: x  =    13; lambda =     0
node 5:    pi =   -12
arc  5->7: x  =     4; lambda =    -1
arc  5->6: x  =    11; lambda =     0
arc  5->2: x  =     0; lambda =     1
node 6:    pi =   -17
arc  6->8: x  =     4; lambda =     3
arc  6->7: x  =     7; lambda =    -3
node 7:    pi =   -20
arc  7->9: x  =    11; lambda =     0
node 8:    pi =   -14
arc  8->9: x  =     9; lambda =     0
node 9:    pi =   -23
\end{verbatim}
\end{footnotesize}

\subsection{glp\_mincost\_relax4 --- solve minimum cost flow problem
with relaxation\\method of Bertsekas and Tseng (RELAX-IV)}

\synopsis

\begin{verbatim}
   int glp_mincost_relax4(glp_graph *G, int v_rhs, int a_low, int a_cap,
                 int a_cost, int crash, double *sol, int a_x, int a_rc);
\end{verbatim}

\description

The routine \verb|glp_mincost_relax4| finds optimal solution to the
minimum cost flow problem with the relaxation method RELAX-IV developed
by Bertsekas and Tseng.\footnote{GLPK implementation of this method is
based on a C translation of the original Fortran code {\tt RELAX4}
written by Dimitri P. Bertsekas and Paul Tseng, with a contribution by
Jonathan Eckstein in the phase II initialization.} This method is one
of most efficient methods for network optimization.

Note that this routine requires all the problem data to be
integer-valued.

The parameter \verb|G| is a graph (network) program object which
specifies the minimum cost flow problem instance to be solved.

The parameter \verb|v_rhs| specifies an offset of the field of type
\verb|double| in the vertex data block, which contains $b_i$, the
supply/demand value. This value must be integer in the range
[$-$\verb|INT_MAX|/4, $+$\verb|INT_MAX|/4]. If \verb|v_rhs| $<0$, it is
assumed that $b_i=0$ for all nodes.

The parameter \verb|a_low| specifies an offset of the field of type
\verb|double| in the arc data block, which contains $l_{ij}$, the lower
bound to the arc flow. This bound must be integer in the range
{\linebreak} [$0$, \verb|INT_MAX|/4]. If \verb|a_low| $<0$, it is
assumed that $l_{ij}=0$ for all arcs.

The parameter \verb|a_cap| specifies an offset of the field of type
\verb|double| in the arc data block, which contains $u_{ij}$, the upper
bound to the arc flow (the arc capacity). This bound must be integer in
the range [$l_{ij}$, \verb|INT_MAX|/4]. If \verb|a_cap| $<0$, it is
assumed that $u_{ij}=1$ for all arcs.

The parameter \verb|a_cost| specifies an offset of the field of type
\verb|double| in the arc data block, which contains $c_{ij}$, the
per-unit cost of the arc flow. This value must be integer in the range
[$-$\verb|INT_MAX|/4, $+$\verb|INT_MAX|/4]. If \verb|a_cost| $<0$, it
is assumed that $c_{ij}=0$ for all arcs.

The parameter \verb|crash| is option specifying initialization method:

0 --- default initialization is used;

1 --- auction initialization is used.

\noindent
If \verb|crash| = 1, initialization is performed with a special crash
procedure based on an auction/shorest path method. This option is
recommended for difficult problems where the default initialization
results in long running times.

The parameter \verb|sol| specifies a location, to which the routine
stores the objective value (that is, the total cost) found. If
\verb|sol| is NULL, the objective value is not stored.

\newpage

The parameter \verb|a_x| specifies an offset of the field of type
\verb|double| in the arc data block, to which the routine stores
$x_{ij}$, the arc flow found. If \verb|a_x| $<0$, the arc flow value is
not stored.

The parameter \verb|a_rc| specifies an offset of the field of type
\verb|double| in the arc data block, to which the routine stores
the reduced cost for corresponding arc flow (see (3) in Subsection
``Background''). If \verb|a_rc| $<0$, the reduced cost is not stored.

Note that all solution components (the objective value, arc flows, and
node potentials) computed by the routine are always integer-valued.

\returns

\begin{retlist}
0 & Optimal solution found.\\

\verb|GLP_ENOPFS| & No (primal) feasible solution exists.\\

\verb|GLP_EDATA| & Unable to start the search, because some problem
data are either not integer-valued or out of range.\\

\verb|GLP_ERANGE| & Unable to start the search because of integer
overflow.\\
\end{retlist}

\para{Example}

The example program below reads the minimum cost problem instance in
DIMACS format from file `\verb|sample.min|', solves it by using the
routine \verb|glp_mincost_relax4|, and writes the solution found on the
standard output.

\begin{footnotesize}
\begin{verbatim}
#include <stddef.h>
#include <stdio.h>
#include <stdlib.h>
#include <glpk.h>

typedef struct { double rhs; } v_data;
typedef struct { double low, cap, cost, x, rc; } a_data;

#define node(v) ((v_data *)((v)->data))
#define arc(a)  ((a_data *)((a)->data))

int main(void)
{     glp_graph *G;
      glp_vertex *v, *w;
      glp_arc *a;
      int i, ret;
      double sol;
      G = glp_create_graph(sizeof(v_data), sizeof(a_data));
      glp_read_mincost(G, offsetof(v_data, rhs),
         offsetof(a_data, low), offsetof(a_data, cap),
         offsetof(a_data, cost), "sample.min");
      ret = glp_mincost_relax4(G, offsetof(v_data, rhs),
         offsetof(a_data, low), offsetof(a_data, cap),
         offsetof(a_data, cost), 0, &sol, offsetof(a_data, x),
         offsetof(a_data, rc));
      printf("ret = %d; sol = %5g\n", ret, sol);
      for (i = 1; i <= G->nv; i++)
      {  v = G->v[i];
         for (a = v->out; a != NULL; a = a->t_next)
         {  w = a->head;
            printf("arc %d->%d: x = %5g; rc = %5g\n",
               v->i, w->i, arc(a)->x, arc(a)->rc);
         }
      }
      glp_delete_graph(G);
      return 0;
}
\end{verbatim}
\end{footnotesize}

If `\verb|sample.min|' is the example data file from the subsection
describing \verb|glp_read_mincost|, the output may look like follows:

\begin{footnotesize}
\begin{verbatim}
Reading min-cost flow problem data from `sample.min'...
Flow network has 9 nodes and 14 arcs
24 lines were read
ret = 0; sol =   213
arc 1->4: x =    13; rc =     0
arc 1->2: x =     7; rc =     0
arc 2->4: x =     0; rc =     3
arc 2->3: x =     7; rc =     0
arc 3->8: x =     5; rc =     0
arc 3->5: x =     2; rc =     3
arc 4->5: x =    13; rc =     0
arc 5->7: x =     4; rc =    -1
arc 5->6: x =    11; rc =     0
arc 5->2: x =     0; rc =     1
arc 6->8: x =     4; rc =     3
arc 6->7: x =     7; rc =    -3
arc 7->9: x =    11; rc =     0
arc 8->9: x =     9; rc =     0
\end{verbatim}
\end{footnotesize}

\subsection{glp\_netgen --- Klingman's network problem generator}

\synopsis

\begin{verbatim}
   int glp_netgen(glp_graph *G, int v_rhs, int a_cap, int a_cost,
                  const int parm[1+15]);
\end{verbatim}

\description

The routine \verb|glp_netgen| is a GLPK version of the network problem
generator developed by Dr.~Darwin~Klingman.\footnote{D.~Klingman,
A.~Napier, and J.~Stutz. NETGEN: A program for generating large scale
capacitated assignment, transportation, and minimum cost flow networks.
Management Science 20 (1974), 814-20.} It can create capacitated and
uncapacitated minimum cost flow (or transshipment), transportation, and
assignment problems.

The parameter \verb|G| specifies the graph object, to which the
generated  problem data have to be stored. Note that on entry the graph
object  is erased with the routine \verb|glp_erase_graph|.

The parameter \verb|v_rhs| specifies an offset of the field of type
\verb|double| in the vertex data block, to which the routine stores the
supply or  demand value. If \verb|v_rhs| $<0$, the value is not stored.

The parameter \verb|a_cap| specifies an offset of the field of type
\verb|double| in the arc data block, to which the routine stores the
arc capacity. If \verb|a_cap| $<0$, the capacity is not stored.

\newpage

The parameter \verb|a_cost| specifies an offset of the field of type
\verb|double| in the arc data block, to which the routine stores the
per-unit cost if the arc flow. If \verb|a_cost| $<0$, the cost is not
stored.

The array \verb|parm| contains description of the network to be
generated:

\begin{tabular}{@{}lll@{}}
\verb|parm[0] |&             &not used\\
\verb|parm[1] |&\verb|iseed |&8-digit positive random number seed\\
\verb|parm[2] |&\verb|nprob |&8-digit problem id number\\
\verb|parm[3] |&\verb|nodes |&total number of nodes\\
\verb|parm[4] |&\verb|nsorc |&total number of source nodes
(including transshipment nodes)\\
\verb|parm[5] |&\verb|nsink |&total number of sink nodes
(including transshipment nodes)\\
\verb|parm[6] |&\verb|iarcs |&number of arc\\
\verb|parm[7] |&\verb|mincst|&minimum cost for arcs\\
\verb|parm[8] |&\verb|maxcst|&maximum cost for arcs\\
\verb|parm[9] |&\verb|itsup |&total supply\\
\verb|parm[10]|&\verb|ntsorc|&number of transshipment source nodes\\
\verb|parm[11]|&\verb|ntsink|&number of transshipment sink nodes\\
\verb|parm[12]|&\verb|iphic |&percentage of skeleton arcs to be given
the maximum cost\\
\verb|parm[13]|&\verb|ipcap |&percentage of arcs to be capacitated\\
\verb|parm[14]|&\verb|mincap|&minimum upper bound for capacitated arcs\\
\verb|parm[15]|&\verb|maxcap|&maximum upper bound for capacitated arcs\\
\end{tabular}

\returns

If the instance was successfully generated, the routine
\verb|glp_netgen| returns zero; otherwise, if specified parameters are
inconsistent, the routine returns a non-zero error code.

\para{Notes}

1. The routine generates a transportation problem if:
$${\tt nsorc}+{\tt nsink}={\tt nodes},
\  {\tt ntsorc}=0,\ \mbox{and}\ {\tt ntsink}=0.$$

2. The routine generates an assignment problem if the requirements for
a transportation problem are met and:
$${\tt nsorc}={\tt nsink}\ \mbox{and}\ {\tt itsup}={\tt nsorc}.$$

3. The routine always generates connected graphs. So, if the number of
requested arcs has been reached and the generated instance is not fully
connected, the routine generates a few remaining arcs to ensure
connectedness. Thus, the actual number of arcs generated by the routine
may be greater than the requested number of arcs.

\subsection{glp\_netgen\_prob --- Klingman's standard network problem
instance}

\synopsis

\begin{verbatim}
   void glp_netgen_prob(int nprob, int parm[1+15]);
\end{verbatim}

\description

The routine \verb|glp_netgen_prob| provides the set of parameters for
Klingman's network problem generator (see the routine
\verb|glp_netgen|), which describe a standard network problem instance.

The parameter \verb|nprob| ($101\leq$ \verb|nprob| $\leq 150$)
specifies the problem instance number.

The array \verb|parm| contains description of the network, provided by
the routine. (For detailed description of these parameters see comments
to the routine \verb|glp_netgen|.)

\para{Problem characteristics}

The table below shows characteristics of Klingman's standard network
problem instances.
$$
\begin{array}{crrr}
{\rm Problem} & {\rm Nodes} & {\rm Arcs} & {\rm Optimum} \\
\hline
101     & 5000  & 25336 &     6191726 \\
102     & 5000  & 25387 &    72337144 \\
103     & 5000  & 25355 &   218947553 \\
104     & 5000  & 25344 &   -19100371 \\
105     & 5000  & 25332 &    31192578 \\
106     & 5000  & 12870 &     4314276 \\
107     & 5000  & 37832 &     7393769 \\
108     & 5000  & 50309 &     8405738 \\
109     & 5000  & 75299 &     9190300 \\
110     & 5000  & 12825 &     8975048 \\
111     & 5000  & 37828 &     4747532 \\
112     & 5000  & 50325 &     4012671 \\
113     & 5000  & 75318 &     2979725 \\
114     & 5000  & 26514 &     5821181 \\
115     & 5000  & 25962 &     6353310 \\
116     & 5000  & 25304 &     5915426 \\
117     & 5000  & 12816 &     4420560 \\
118     & 5000  & 37797 &     7045842 \\
119     & 5000  & 50301 &     7724179 \\
120     & 5000  & 75330 &     8455200 \\
121     & 5000  & 25000 &    66366360 \\
122     & 5000  & 25000 &    30997529 \\
123     & 5000  & 25000 &    23388777 \\
124     & 5000  & 25000 &    17803443 \\
125     & 5000  & 25000 &    14119622 \\
\end{array}
\hspace{.5in}
\begin{array}{crrr}
{\rm Problem} & {\rm Nodes} & {\rm Arcs} & {\rm Optimum} \\
\hline
126     & 5000  & 12500 &    18802218 \\
127     & 5000  & 37500 &    27674647 \\
128     & 5000  & 50000 &    30906194 \\
129     & 5000  & 75000 &    40905209 \\
130     & 5000  & 12500 &    38939608 \\
131     & 5000  & 37500 &    16752978 \\
132     & 5000  & 50000 &    13302951 \\
133     & 5000  & 75000 &     9830268 \\
134     & 1000  & 25000 &     3804874 \\
135     & 2500  & 25000 &    11729616 \\
136     & 7500  & 25000 &    33318101 \\
137    & 10000  & 25000 &    46426030 \\
138     & 5000  & 25000 &    60710879 \\
139     & 5000  & 25000 &    32729682 \\
140     & 5000  & 25000 &    27183831 \\
141     & 5000  & 25000 &    19963286 \\
142     & 5000  & 25000 &    20243457 \\
143     & 5000  & 25000 &    18586777 \\
144     & 5000  & 25000 &     2504591 \\
145     & 5000  & 25000 &   215956138 \\
146     & 5000  & 25000 &  2253113811 \\
147     & 5000  & 25000 &  -427908373 \\
148     & 5000  & 25000 &   -92965318 \\
149     & 5000  & 25000 &    86051224 \\
150     & 5000  & 25000 &   619314919 \\
\end{array}
$$

\subsection{glp\_gridgen --- grid-like network problem generator}

\synopsis

\begin{verbatim}
   int glp_gridgen(glp_graph *G, int v_rhs, int a_cap, int a_cost,
                   const int parm[1+14]);
\end{verbatim}

\description

The routine \verb|glp_gridgen| is a GLPK version of the grid-like
network problem generator developed by Yusin Lee and Jim
Orlin.\footnote{Y.~Lee and J.~Orlin. GRIDGEN generator., 1991. The
original code is publically available from
\url{ftp://dimacs.rutgers.edu/pub/netflow/generators/network/gridgen}.}

\newpage

The parameter \verb|G| specifies the graph object, to which the
generated  problem data have to be stored. Note that on entry the graph
object  is erased with the routine \verb|glp_erase_graph|.

The parameter \verb|v_rhs| specifies an offset of the field of type
\verb|double| in the vertex data block, to which the routine stores the
supply or  demand value. If \verb|v_rhs| $<0$, the value is not stored.

The parameter \verb|a_cap| specifies an offset of the field of type
\verb|double| in the arc data block, to which the routine stores the
arc capacity. If \verb|a_cap| $<0$, the capacity is not stored.

The parameter \verb|a_cost| specifies an offset of the field of type
\verb|double| in the arc data block, to which the routine stores the
per-unit cost if the arc flow. If \verb|a_cost| $<0$, the cost is not
stored.

The array \verb|parm| contains parameters of the network to be
generated:

\begin{tabular}{@{}ll@{}}
\verb|parm[0] |&not used\\
\verb|parm[1] |&two-ways arcs indicator:\\
               &1 --- if links in both direction should be generated\\
               &0 --- otherwise\\
\verb|parm[2] |&random number seed (a positive integer)\\
\verb|parm[3] |&number of nodes (the number of nodes generated might
be slightly different to\\&make the network a grid)\\
\verb|parm[4] |&grid width\\
\verb|parm[5] |&number of sources\\
\verb|parm[6] |&number of sinks\\
\verb|parm[7] |&average degree\\
\verb|parm[8] |&total flow\\
\verb|parm[9] |&distribution of arc costs:
1 --- uniform, 2 --- exponential\\
\verb|parm[10]|&lower bound for arc cost (uniform),
$100\lambda$ (exponential)\\
\verb|parm[11]|&upper bound for arc cost (uniform),
not used (exponential)\\
\verb|parm[12]|&distribution of arc capacities:
1 --- uniform, 2 --- exponential\\
\verb|parm[13]|&lower bound for arc capacity (uniform),
$100\lambda$ (exponential)\\
\verb|parm[14]|&upper bound for arc capacity (uniform),
not used (exponential)\\
\end{tabular}

\returns

If the instance was successfully generated, the routine
\verb|glp_gridgen| returns zero; otherwise, if specified parameters are
inconsistent, the routine returns a non-zero error code.

\para{Comments\footnote{This material is based on comments
to the original version of GRIDGEN.}}

This network generator generates a grid-like network plus a super node.
In additional to the arcs connecting the nodes in the grid, there is an
arc from each supply node to the super node and from the super node to
each demand node to guarantee feasiblity. These arcs have very high
costs and very big capacities.

The idea of this network generator is as follows: First, a grid of
$n_1\times n_2$ is generated. For example, $5\times 3$. The nodes are
numbered as 1 to 15, and the supernode is numbered as
$n_1\times n_2+1$. Then arcs between adjacent nodes are generated.
For these arcs, the user is allowed to specify either to generate
two-way arcs or one-way arcs. If two-way arcs are to be generated, two
arcs, one in each direction, will be generated between each adjacent
node pairs. Otherwise, only one arc will be generated. If this is the
case, the arcs will be generated in alterntive directions as shown
below.

\medskip

\noindent\hfil
\xymatrix
{1\ar[r]\ar[d]&2\ar[r]&3\ar[r]\ar[d]&4\ar[r]&5\ar[d]\\
6\ar[d]&7\ar[l]\ar[u]&8\ar[l]\ar[d]&9\ar[l]\ar[u]&10\ar[l]\ar[d]\\
11\ar[r]&12\ar[r]\ar[u]&13\ar[r]&14\ar[r]\ar[u]&15\\
}

\medskip

Then the arcs between the super node and the source/sink nodes are
added as mentioned before. If the number of arcs still doesn't reach
the requirement, additional arcs will be added by uniformly picking
random node pairs. There is no checking to prevent multiple arcs
between any pair of nodes. However, there will be no self-arcs (arcs
that poins back to its tail node) in the network.

The source and sink nodes are selected uniformly in the network, and
the imbalances of each source/sink node are also assigned by uniform
distribution.

%%%%%%%%%%%%%%%%%%%%%%%%%%%%%%%%%%%%%%%%%%%%%%%%%%%%%%%%%%%%%%%%%%%%%%%%

\newpage

\section{Maximum flow problem}

\subsection{Background}

The {\it maximum flow problem} (MAXFLOW) is stated as follows. Let
there be given a directed graph (flow network) $G=(V,A)$, where $V$ is
a set of vertices (nodes), and $A\subseteq V\times V$ is a set of arcs.
Let also for each arc $a=(i,j)\in A$ there be given its capacity
$u_{ij}$. The problem is, for given {\it source} node $s\in V$ and
{\it sink} node $t\in V$, to find flows $x_{ij}$ through every arc of
the network, which satisfy the specified arc capacities and the
conservation constraints at all nodes, and maximize the total flow $F$
through the network from $s$ to $t$. Here the conservation constraint
at a node means that the total flow entering this node through its
incoming arcs (plus $F$, if it is the source node) must be equal to the
total flow leaving this node through its outgoing arcs (plus $F$, if it
is the sink node). An example of the maximum flow problem,
where $s=v_1$ and $t=v_9$, is shown on Fig.~2.

\medskip

\noindent\hfil
\xymatrix @C=48pt
{_{F}\ar@{~>}[d]&
v_2\ar[r]|{_{10}}\ar[dd]|{_{9}}&
v_3\ar[dd]|{_{12}}\ar[r]|{_{18}}&
v_8\ar[rd]|{_{20}}&\\
v_1\ar[ru]|{_{14}}\ar[rd]|{_{23}}&&&
v_6\ar[d]|{_{7}}\ar[u]|{_{8}}&
v_9\ar@{~>}[d]\\
&v_4\ar[r]|{_{26}}&
v_5\ar[luu]|{_{11}}\ar[ru]|{_{25}}\ar[r]|{_{4}}&
v_7\ar[ru]|{_{15}}&_{F}\\
}

\medskip

\noindent\hfil
Fig.~2. An example of the maximum flow problem.

\medskip

The maximum flow problem can be naturally formulated as the following
LP problem:

\noindent
\hspace{1in}maximize
$$F\eqno(4)$$
\hspace{1in}subject to
$$\sum_{(i,j)\in A}x_{ij}-\sum_{(j,i)\in A}x_{ji}=\left\{
\begin{array}{@{\ }rl}
+F,&\hbox{for}\ i=s\\
 0,&\hbox{for all}\ i\in V\backslash\{s,t\}\\
-F,&\hbox{for}\ i=t\\
\end{array}
\right.\eqno(5)
$$
$$0\leq x_{ij}\leq u_{ij}\ \ \ \hbox{for all}\ (i,j)\in A
\eqno(6)$$

\noindent
where $F\geq 0$ is an additional variable playing the role of the
objective.

Another LP formulation of the maximum flow problem, which does not
include the variable $F$, is the following:

\noindent
\hspace{1in}maximize
$$z=\sum_{(s,j)\in A}x_{sj}-\sum_{(j,s)\in A}x_{js}\ (=F)\eqno(7)$$
\hspace{1in}subject to
$$\sum_{(i,j)\in A}x_{ij}-\sum_{(j,i)\in A}x_{ji}\left\{
\begin{array}{@{\ }rl}
\geq 0,&\hbox{for}\ i=s\\
=0,&\hbox{for all}\ i\in V\backslash\{s,t\}\\
\leq 0,&\hbox{for}\ i=t\\
\end{array}
\right.\eqno(8)
$$
$$0\leq x_{ij}\leq u_{ij}\ \ \ \hbox{for all}\ (i,j)\in A
\eqno(9)$$

\subsection{glp\_read\_maxflow --- read maximum flow problem data in
DIMACS\\format}

\synopsis

\begin{verbatim}
   int glp_read_maxflow(glp_graph *G, int *s, int *t, int a_cap,
                        const char *fname);
\end{verbatim}

\description

The routine \verb|glp_read_maxflow| reads the maximum flow problem
data from a text file in DIMACS format.

The parameter \verb|G| specifies the graph object, to which the problem
data have to be stored. Note that before reading data the current
content of the graph object is completely erased with the routine
\verb|glp_erase_graph|.

The pointer \verb|s| specifies a location, to which the routine stores
the ordinal number of the source node. If \verb|s| is \verb|NULL|, the
source node number is not stored.

The pointer \verb|t| specifies a location, to which the routine stores
the ordinal number of the sink node. If \verb|t| is \verb|NULL|, the
sink node number is not stored.

The parameter \verb|a_cap| specifies an offset of the field of type
\verb|double| in the arc data block, to which the routine stores
$u_{ij}$, the arc capacity. If \verb|a_cap| $<0$, the arc capacity is
not stored.

The character string \verb|fname| specifies the name of a text file to
be read in. (If the file name name ends with the suffix `\verb|.gz|',
the file is assumed to be compressed, in which case the routine
decompresses it ``on the fly''.)

\returns

If the operation was successful, the routine returns zero. Otherwise,
it prints an error message and returns non-zero.

\para{Example}

\begin{footnotesize}
\begin{verbatim}
typedef struct
{     /* arc data block */
      ...
      double cap;
      ...
} a_data;

int main(void)
{     glp_graph *G;
      int s, t, ret;
      G = glp_create_graph(..., sizeof(a_data));
      ret = glp_read_maxflow(G, &s, &t, offsetof(a_data, cap),
         "sample.max");
      if (ret != 0) goto ...
      ...
}
\end{verbatim}
\end{footnotesize}

\newpage

\para{DIMACS maximum flow problem format\footnote{This material is
based on the paper ``The First DIMACS International Algorithm
Implementation Challenge: Problem Definitions and Specifications'',
which is publically available at
\url{http://dimacs.rutgers.edu/Challenges/}.}}
\label{subsecmaxflow}

The DIMACS input file is a plain ASCII text file. It contains
{\it lines} of several types described below. A line is terminated with
an end-of-line character. Fields in each line are separated by at least
one blank space. Each line begins with a one-character designator to
identify the line type.

Note that DIMACS requires all numerical quantities to be integers in
the range $[-2^{31},\ 2^{31}-1]$ while GLPK allows the quantities to be
floating-point numbers.

\para{Comment lines.} Comment lines give human-readable information
about the file and are ignored by programs. Comment lines can appear
anywhere in the file. Each comment line begins with a lower-case
character \verb|c|.

\begin{verbatim}
   c This is a comment line
\end{verbatim}

\para{Problem line.} There is one problem line per data file. The
problem line must appear before any node or arc descriptor lines.
It has the following format:

\begin{verbatim}
   p max NODES ARCS
\end{verbatim}

\noindent
The lower-case character \verb|p| signifies that this is a problem line.
The three-character problem designator \verb|max| identifies the file as
containing specification information for the maximum flow problem. The
\verb|NODES| field contains an integer value specifying the number of
nodes in the network. The \verb|ARCS| field contains an integer value
specifying the number of arcs in the network.

\para{Node descriptors.} Two node descriptor lines for the source and
sink nodes must appear before all arc descriptor lines. They may appear
in either order, each with the following format:

\begin{verbatim}
   n ID WHICH
\end{verbatim}

\noindent
The lower-case character \verb|n| signifies that this a node descriptor
line. The \verb|ID| field gives a node identification number,
an integer between 1 and \verb|NODES|. The \verb|WHICH| field gives
either a lower-case \verb|s| or \verb|t|, designating the source and
sink, respectively.

\para{Arc descriptors.} There is one arc descriptor line for each arc
in the network. Arc descriptor lines are of the following format:

\begin{verbatim}
   a SRC DST CAP
\end{verbatim}

\noindent
The lower-case character \verb|a| signifies that this is an arc
descriptor line. For a directed arc $(i,j)$ the \verb|SRC| field gives
the identification number $i$ for the tail endpoint, and the \verb|DST|
field gives the identification number $j$ for the head endpoint.
Identification numbers are integers between 1 and \verb|NODES|. The
\verb|CAP| field gives the arc capacity, i.e. maximum amount of flow
that can be sent along arc $(i,j)$ in a feasible flow.

\para{Example.} Below here is an example of the data file in DIMACS
format corresponding to the maximum flow problem shown on Fig~2.

\begin{footnotesize}
\begin{verbatim}
c sample.max
c
c This is an example of the maximum flow problem data
c in DIMACS format.
c
p max 9 14
c
n 1 s
n 9 t
c
a 1 2 14
a 1 4 23
a 2 3 10
a 2 4  9
a 3 5 12
a 3 8 18
a 4 5 26
a 5 2 11
a 5 6 25
a 5 7  4
a 6 7  7
a 6 8  8
a 7 9 15
a 8 9 20
c
c eof
\end{verbatim}
\end{footnotesize}

\subsection{glp\_write\_maxflow --- write maximum flow problem data in
DIMACS\\format}

\synopsis

\begin{verbatim}
   int glp_write_maxflow(glp_graph *G, int s, int t, int a_cap,
                         const char *fname);
\end{verbatim}

\description

The routine \verb|glp_write_maxflow| writes the maximum flow problem
data to a text file in DIMACS format.

The parameter \verb|G| is the graph (network) program object, which
specifies the maximum flow problem instance.

The parameter \verb|s| specifies the ordinal number of the source node.

The parameter \verb|t| specifies the ordinal number of the sink node.

The parameter \verb|a_cap| specifies an offset of the field of type
\verb|double| in the arc data block, which contains $u_{ij}$, the upper
bound to the arc flow (the arc capacity). If the upper bound is
specified as \verb|DBL_MAX|, it is assumed that $u_{ij}=\infty$, i.e.
the arc is uncapacitated. If \verb|a_cap| $<0$, it is assumed that
$u_{ij}=1$ for all arcs.

The character string \verb|fname| specifies a name of the text file to
be written out. (If the file name ends with suffix `\verb|.gz|', the
file is assumed to be compressed, in which case the routine performs
automatic compression on writing it.)

\returns

If the operation was successful, the routine returns zero. Otherwise,
it prints an error message and returns non-zero.

\newpage

\subsection{glp\_maxflow\_lp --- convert maximum flow problem to LP}

\synopsis

\begin{verbatim}
   void glp_maxflow_lp(glp_prob *P, glp_graph *G, int names, int s, int t,
                       int a_cap);
\end{verbatim}

\description

The routine \verb|glp_maxflow_lp| builds LP problem (7)---(9), which
corresponds to the specified maximum flow problem.

The parameter \verb|P| is the resultant LP problem object to be built.
Note that on entry its current content is erased with the routine
\verb|glp_erase_prob|.

The parameter \verb|G| is the graph (network) program object, which
specifies the maximum flow problem instance.

The parameter \verb|names| is a flag. If it is \verb|GLP_ON|, the
routine uses symbolic names of the graph object components to assign
symbolic names to the LP problem object components. If the flag is
\verb|GLP_OFF|, no symbolic names are assigned.

The parameter \verb|s| specifies the ordinal number of the source node.

The parameter \verb|t| specifies the ordinal number of the sink node.

The parameter \verb|a_cap| specifies an offset of the field of type
\verb|double| in the arc data block, which contains $u_{ij}$, the upper
bound to the arc flow (the arc capacity). If the upper bound is
specified as \verb|DBL_MAX|, it is assumed that $u_{ij}=\infty$, i.e.
the arc is uncapacitated. If \verb|a_cap| $<0$, it is assumed that
$u_{ij}=1$ for all arcs.

\para{Example}

The example program below reads the maximum flow problem in DIMACS
format from file `\verb|sample.max|', converts the instance to LP, and
then writes the resultant LP in CPLEX format to file
`\verb|maxflow.lp|'.

\begin{footnotesize}
\begin{verbatim}
#include <stddef.h>
#include <glpk.h>

int main(void)
{     glp_graph *G;
      glp_prob *P;
      int s, t;
      G = glp_create_graph(0, sizeof(double));
      glp_read_maxflow(G, &s, &t, 0, "sample.max");
      P = glp_create_prob();
      glp_maxflow_lp(P, G, GLP_ON, s, t, 0);
      glp_delete_graph(G);
      glp_write_lp(P, NULL, "maxflow.lp");
      glp_delete_prob(P);
      return 0;
}
\end{verbatim}
\end{footnotesize}

If `\verb|sample.max|' is the example data file from the previous
subsection, the output `\verb|maxflow.lp|' may look like follows:

\newpage

\begin{footnotesize}
\begin{verbatim}
Maximize
 obj: + x(1,4) + x(1,2)

Subject To
 r_1: + x(1,2) + x(1,4) >= 0
 r_2: - x(5,2) + x(2,3) + x(2,4) - x(1,2) = 0
 r_3: + x(3,5) + x(3,8) - x(2,3) = 0
 r_4: + x(4,5) - x(2,4) - x(1,4) = 0
 r_5: + x(5,2) + x(5,6) + x(5,7) - x(4,5) - x(3,5) = 0
 r_6: + x(6,7) + x(6,8) - x(5,6) = 0
 r_7: + x(7,9) - x(6,7) - x(5,7) = 0
 r_8: + x(8,9) - x(6,8) - x(3,8) = 0
 r_9: - x(8,9) - x(7,9) <= 0

Bounds
 0 <= x(1,4) <= 23
 0 <= x(1,2) <= 14
 0 <= x(2,4) <= 9
 0 <= x(2,3) <= 10
 0 <= x(3,8) <= 18
 0 <= x(3,5) <= 12
 0 <= x(4,5) <= 26
 0 <= x(5,7) <= 4
 0 <= x(5,6) <= 25
 0 <= x(5,2) <= 11
 0 <= x(6,8) <= 8
 0 <= x(6,7) <= 7
 0 <= x(7,9) <= 15
 0 <= x(8,9) <= 20

End
\end{verbatim}
\end{footnotesize}

\subsection{glp\_maxflow\_ffalg --- solve maximum flow problem with
Ford-Fulkerson\\algorithm}

\synopsis

\begin{verbatim}
   int glp_maxflow_ffalg(glp_graph *G, int s, int t, int a_cap, double *sol,
                         int a_x, int v_cut);
\end{verbatim}

\description

The routine \verb|glp_mincost_ffalg| finds optimal solution to the
maximum flow problem with the Ford-Fulkerson algorithm.\footnote{GLPK
implementation of the Ford-Fulkerson algorithm is based on the
following book: L.~R.~Ford,~Jr., and D.~R.~Fulkerson, ``Flows in
Networks,'' The RAND Corp., Report R-375-PR (August 1962), Chap. I
``Static Maximal Flow,'' pp.~30-33.} Note that this routine requires
all the problem data to be integer-valued.

The parameter \verb|G| is a graph (network) program object which
specifies the maximum flow problem instance to be solved.

The parameter $s$ specifies the ordinal number of the source node.

\newpage

The parameter $t$ specifies the ordinal number of the sink node.

The parameter \verb|a_cap| specifies an offset of the field of type
\verb|double| in the arc data block, which contains $u_{ij}$, the upper
bound to the arc flow (the arc capacity). This bound must be integer in
the range [0, \verb|INT_MAX|]. If \verb|a_cap| $<0$, it is assumed that
$u_{ij}=1$ for all arcs.

The parameter \verb|sol| specifies a location, to which the routine
stores the objective value (that is, the total flow from $s$ to $t$)
found. If \verb|sol| is NULL, the objective value is not stored.

The parameter \verb|a_x| specifies an offset of the field of type
\verb|double| in the arc data block, to which the routine stores
$x_{ij}$, the arc flow found. If \verb|a_x| $<0$, the arc flow values
are not stored.

The parameter \verb|v_cut| specifies an offset of the field of type
\verb|int| in the vertex data block, to which the routine stores node
flags corresponding to the optimal solution found: if the node flag is
1, the node is labelled, and if the node flag is 0, the node is
unlabelled. The calling program may use these node flags to determine
the {\it minimal cut}, which is a subset of arcs whose one endpoint is
labelled and other is not. If \verb|v_cut| $<0$, the node flags are not
stored.

Note that all solution components (the objective value and arc flows)
computed by the routine are always integer-valued.

\returns

\begin{retlist}
0 & Optimal solution found.\\

\verb|GLP_EDATA| & Unable to start the search, because some problem
data are either not integer-valued or out of range.\\
\end{retlist}

\para{Example}

The example program shown below reads the maximum flow problem instance
in DIMACS format from file `\verb|sample.max|', solves it using the
routine \verb|glp_maxflow_ffalg|, and write the solution found to the
standard output.

\begin{footnotesize}
\begin{verbatim}
#include <stddef.h>
#include <stdio.h>
#include <stdlib.h>
#include <glpk.h>

typedef struct { int cut; } v_data;
typedef struct { double cap, x; } a_data;

#define node(v) ((v_data *)((v)->data))
#define arc(a)  ((a_data *)((a)->data))

int main(void)
{     glp_graph *G;
      glp_vertex *v, *w;
      glp_arc *a;
      int i, s, t, ret;
      double sol;
      G = glp_create_graph(sizeof(v_data), sizeof(a_data));
      glp_read_maxflow(G, &s, &t, offsetof(a_data, cap),
         "sample.max");
      ret = glp_maxflow_ffalg(G, s, t, offsetof(a_data, cap),
         &sol, offsetof(a_data, x), offsetof(v_data, cut));
      printf("ret = %d; sol = %5g\n", ret, sol);
      for (i = 1; i <= G->nv; i++)
      {  v = G->v[i];
         for (a = v->out; a != NULL; a = a->t_next)
         {  w = a->head;
            printf("x[%d->%d] = %5g (%d)\n", v->i, w->i,
               arc(a)->x, node(v)->cut ^ node(w)->cut);
         }
      }
      glp_delete_graph(G);
      return 0;
}
\end{verbatim}
\end{footnotesize}

If `\verb|sample.max|' is the example data file from the subsection
describing \verb|glp_read_maxflow|, the output may look like follows:

\begin{footnotesize}
\begin{verbatim}
Reading maximum flow problem data from `sample.max'...
Flow network has 9 nodes and 14 arcs
24 lines were read
ret = 0; sol =    29
x[1->4] =    19 (0)
x[1->2] =    10 (0)
x[2->4] =     0 (0)
x[2->3] =    10 (1)
x[3->8] =    10 (0)
x[3->5] =     0 (1)
x[4->5] =    19 (0)
x[5->7] =     4 (1)
x[5->6] =    15 (0)
x[5->2] =     0 (0)
x[6->8] =     8 (1)
x[6->7] =     7 (1)
x[7->9] =    11 (0)
x[8->9] =    18 (0)
\end{verbatim}
\end{footnotesize}

\subsection{glp\_rmfgen --- Goldfarb's maximum flow problem generator}

\synopsis

\begin{verbatim}
   int glp_rmfgen(glp_graph *G, int *s, int *t, int a_cap, const int parm[1+5]);
\end{verbatim}

\description

The routine \verb|glp_rmfgen| is a GLPK version of the maximum flow
problem generator developed by D.~Goldfarb and
M.~Grigoriadis.\footnote{D.~Goldfarb and M.~D.~Grigoriadis,
``A computational comparison of the Dinic and network simplex methods
for maximum flow.'' Annals of Op. Res. 13 (1988),
pp.~83-123.}$^{,}$\footnote{U.~Derigs and W.~Meier, ``Implementing
Goldberg's max-flow algorithm: A computational investigation.''
Zeitschrift f\"ur Operations Research 33 (1989),
pp.~383-403.}$^{,}$\footnote{The original code of RMFGEN implemented by
Tamas Badics is publically available from
\url{ftp://dimacs.rutgers.edu/pub/netflow/generators/network/genrmf}.}

The parameter \verb|G| specifies the graph object, to which the
generated problem data have to be stored. Note that on entry the graph
object is erased with the routine \verb|glp_erase_graph|.

\newpage

The pointers \verb|s| and \verb|t| specify locations, to which the
routine stores the source and sink node numbers, respectively. If
\verb|s| or \verb|t| is \verb|NULL|, corresponding node number is not
stored.

The parameter \verb|a_cap| specifies an offset of the field of type
\verb|double| in the arc data block, to which the routine stores the arc
capacity. If \verb|a_cap| $<0$, the capacity is not stored.

The array \verb|parm| contains description of the network to be
generated:

\begin{tabular}{@{}lll@{}}
\verb|parm[0]|&           &not used\\
\verb|parm[1]|&\verb|seed|&random number seed (a positive integer)\\
\verb|parm[2]|&\verb|a   |&frame size\\
\verb|parm[3]|&\verb|b   |&depth\\
\verb|parm[4]|&\verb|c1  |&minimal arc capacity\\
\verb|parm[5]|&\verb|c2  |&maximal arc capacity\\
\end{tabular}

\returns

If the instance was successfully generated, the routine
\verb|glp_netgen| returns zero; otherwise, if specified parameters are
inconsistent, the routine returns a non-zero error code.

\para{Comments\footnote{This material is based on comments to the
original version of RMFGEN.}}

The generated network is as follows. It has $b$ pieces of frames of
size $a\times a$. (So alltogether the number of vertices is
$a\times a\times b$.)

In each frame all the vertices are connected with their neighbours
(forth and back). In addition the vertices of a frame are connected
one to one with the vertices of next frame using a random permutation
of those vertices.

The source is the lower left vertex of the first frame, the sink is
the upper right vertex of the $b$-th frame.

\begin{verbatim}
                                             t
                                    +-------+
                                    |      .|
                                    |     . |
                                 /  |    /  |
                                +-------+/ -+ b
                                |    |  |/.
                              a |   -v- |/
                                |    |  |/
                                +-------+ 1
                               s    a
\end{verbatim}

The capacities are randomly chosen integers from the range of
$[c_1,c_2]$  in the case of interconnecting edges, and $c_2\cdot a^2$
for the in-frame edges.

%%%%%%%%%%%%%%%%%%%%%%%%%%%%%%%%%%%%%%%%%%%%%%%%%%%%%%%%%%%%%%%%%%%%%%%%

\newpage

\section{Assignment problem}

\subsection{Background}

Let there be given an undirected bipartite graph $G=(R\cup S,E)$, where
$R$ and $S$ are disjoint sets of vertices (nodes), and
$E\subseteq R\times S$ is a set of edges. Let also for each edge
$e=(i,j)\in E$ there be given its cost $c_{ij}$. A {\it matching}
(which in case of bipartite graph is also called {\it assignment})
$M\subseteq E$ in $G$ is a set of pairwise non-adjacent edges, that is,
no two edges in $M$ share a common vertex. A matching, which matches
all vertices of the graph, is called a {\it perfect matching}.
Obviously, a perfect matching in bipartite graph $G=(R\cup S,E)$
defines some bijection $R\leftrightarrow S$.

The {\it assignment problem} has two different variants. In the first
variant the problem is to find matching (assignment) $M$, which
maximizes the sum:
$$\sum_{(i,j)\in M}c_{ij}\eqno(10)$$
(so this variant is also called the {\it maximum weighted bipartite
matching problem} or, if all $c_{ij}=1$, the {\it maximum cardinality
bipartite matching problem}). In the second, classic variant the
problem is to find {\it perfect} matching (assignment) $M$, which
minimizes or maximizes the sum (10).

An example of the assignment problem, which is the maximum weighted
bipartite matching problem, is shown on Fig. 3.

The maximum weighted bipartite matching problem can be naturally
formulated as the following LP problem:

\noindent
\hspace{1in}maximize
$$z=\sum_{(i,j)\in E}c_{ij}x_{ij}\eqno(11)$$
\hspace{1in}subject to
$$\sum_{(i,j)\in E}x_{ij}\leq 1\ \ \ \hbox{for all}\ i\in R\eqno(12)$$
$$\sum_{(i,j)\in E}x_{ij}\leq 1\ \ \ \hbox{for all}\ j\in S\eqno(13)$$
$$\ \ \ \ \ \ \ \ 0\leq x_{ij}\leq 1\ \ \ \hbox{for all}\ (i,j)\in E
\eqno(14)$$

\noindent
where $x_{ij}=1$ means that $(i,j)\in M$, and $x_{ij}=0$ means that
$(i,j)\notin M$.\footnote{The constraint matrix of LP formulation
(11)---(14) is totally unimodular, due to which $x_{ij}\in\{0,1\}$ for
any basic solution.}

\newpage

\noindent\hfil
\xymatrix @C=48pt
{v_1\ar@{-}[rr]|{_{13}}\ar@{-}[rrd]|{_{21}}\ar@{-}[rrddd]|(.2){_{20}}&&
v_9\\
v_2\ar@{-}[rr]|{_{12}}\ar@{-}[rrdd]|(.3){_{8}}
\ar@{-}[rrddd]|(.4){_{26}}&&v_{10}\\
v_3\ar@{-}[rr]|(.2){_{22}}\ar@{-}[rrdd]|(.3){_{11}}&&v_{11}\\
v_4\ar@{-}[rruuu]|(.6){_{12}}\ar@{-}[rr]|(.2){_{36}}
\ar@{-}[rrdd]|(.7){_{25}}&&v_{12}\\
v_5\ar@{-}[rruu]|(.42){_{41}}\ar@{-}[rru]|(.4){_{40}}
\ar@{-}[rr]|(.75){_{11}}\ar@{-}[rrd]|(.6){_{4}}\ar@{-}[rrdd]|{_{8}}
\ar@{-}[rrddd]|{_{35}}\ar@{-}[rrdddd]|{_{32}}&&v_{13}\\
v_6\ar@{-}[rruuuuu]|(.7){_{13}}&&v_{14}\\
v_7\ar@{-}[rruuuuu]|(.15){_{19}}&&v_{15}\\
v_8\ar@{-}[rruuuuuu]|(.25){_{39}}\ar@{-}[rruuuuu]|(.65){_{15}}&&
v_{16}\\
&&v_{17}\\
}

\medskip

\noindent\hfil
Fig.~3. An example of the assignment problem.

\medskip

Similarly, the perfect assignment problem can be naturally formulated
as the following LP problem:

\noindent
\hspace{1in}minimize (or maximize)
$$z=\sum_{(i,j)\in E}c_{ij}x_{ij}\eqno(15)$$
\hspace{1in}subject to
$$\sum_{(i,j)\in E}x_{ij}=1\ \ \ \hbox{for all}\ i\in R\eqno(16)$$
$$\sum_{(i,j)\in E}x_{ij}=1\ \ \ \hbox{for all}\ j\in S\eqno(17)$$
$$\ \ \ \ \ \ \ \ 0\leq x_{ij}\leq 1\ \ \ \hbox{for all}\ (i,j)\in E
\eqno(18)$$

\noindent
where variables $x_{ij}$ have the same meaning as for (11)---(14)
above.

In GLPK an undirected bipartite graph $G=(R\cup S,E)$ is represented as
directed graph $\overline{G}=(V,A)$, where $V=R\cup S$ and
$A=\{(i,j):(i,j)\in E\}$, i.e. every edge $(i,j)\in E$ in $G$
corresponds to arc $(i\rightarrow j)\in A$ in $\overline{G}$.

\subsection{glp\_read\_asnprob --- read assignment problem data in
DIMACS format}

\synopsis

\begin{verbatim}
   int glp_read_asnprob(glp_graph *G, int v_set, int a_cost, const char *fname);
\end{verbatim}

\description

The routine \verb|glp_read_asnprob| reads the assignment problem data
from a text file in DIMACS format.

The parameter \verb|G| specifies the graph object, to which the problem
data have to be stored. Note that before reading data the current
content of the graph object is completely erased with the routine
\verb|glp_erase_graph|.

The parameter \verb|v_set| specifies an offset of the field of type
\verb|int| in the vertex data block, to which the routine stores the
node set indicator:

0 --- the node is in set $R$;

1 --- the node is in set $S$.

\noindent
If \verb|v_set| $<0$, the node set indicator is not stored.

The parameter \verb|a_cost| specifies an offset of the field of type
\verb|double| in the arc data block, to which the routine stores the
edge cost $c_{ij}$. If \verb|a_cost| $<0$, the edge cost is not stored.

The character string \verb|fname| specifies the name of a text file to
be read in. (If the file name name ends with the suffix `\verb|.gz|',
the file is assumed to be compressed, in which case the routine
decompresses it ``on the fly''.)

\returns

If the operation was successful, the routine returns zero. Otherwise,
it prints an error message and returns non-zero.

\para{Example.} Below here is an example program that read the
assignment problem data in DIMACS format from a text file
`\verb|sample.asn|'.

\begin{footnotesize}
\begin{verbatim}
typedef struct
{     /* vertex data block */
      ...
      int set;
      ...
} v_data;

typedef struct
{     /* arc data block */
      ...
      double cost;
      ...
} a_data;

int main(void)
{     glp_graph *G;
      int ret;
      G = glp_create_graph(sizeof(v_data), sizeof(a_data));
      ret = glp_read_asnprob(G, offsetof(v_data, set),
         offsetof(a_data, cost), "sample.asn");
      if (ret != 0) goto ...
      ...
}
\end{verbatim}
\end{footnotesize}

\para{DIMACS assignment problem format\footnote{This material is based
on the paper ``The First DIMACS International Algorithm Implementation
Challenge: Problem Definitions and Specifications'', which is
publically available at \url{http://dimacs.rutgers.edu/Challenges/}.}}
\label{subsecasnprob}

The DIMACS input file is a plain ASCII text file. It contains
{\it lines} of several types described below. A line is terminated with
an end-of-line character. Fields in each line are separated by at least
one blank space. Each line begins with a one-character designator to
identify the line type.

Note that DIMACS requires all numerical quantities to be integers in
the range $[-2^{31},\ 2^{31}-1]$ while GLPK allows the quantities to be
floating-point numbers.

\para{Comment lines.} Comment lines give human-readable information
about the file and are ignored by programs. Comment lines can appear
anywhere in the file. Each comment line begins with a lower-case
character \verb|c|.

\begin{verbatim}
   c This is a comment line
\end{verbatim}

\para{Problem line.} There is one problem line per data file. The
problem line must appear before any node or arc descriptor lines. It
has the following format:

\begin{verbatim}
   p asn NODES EDGES
\end{verbatim}

\noindent
The lower-case character \verb|p| signifies that this is a problem line.
The three-character problem designator \verb|asn| identifies the file as
containing specification information for the assignment problem.
The \verb|NODES| field contains an integer value specifying the total
number of nodes in the graph (i.e. in both sets $R$ and $S$). The
\verb|EDGES| field contains an integer value specifying the number of
edges in the graph.

\para{Node descriptors.} All node descriptor lines must appear before
all edge descriptor lines. The node descriptor lines lists the nodes in
set $R$ only, and all other nodes are assumed to be in set $S$. There
is one node descriptor line for each such node, with the following
format:

\begin{verbatim}
   n ID
\end{verbatim}

\noindent
The lower-case character \verb|n| signifies that this is a node
descriptor line. The \verb|ID| field gives a node identification number,
an integer between 1 and \verb|NODES|.

\para{Edge descriptors.} There is one edge descriptor line for each
edge in the graph. Edge descriptor lines are of the following format:

\begin{verbatim}
   a SRC DST COST
\end{verbatim}

\noindent
The lower-case character \verb|a| signifies that this is an edge
descriptor line. For each edge $(i,j)$, where $i\in R$ and $j\in S$,
the \verb|SRC| field gives the identification number of vertex $i$, and
the \verb|DST| field gives the identification number of vertex $j$.
Identification numbers are integers between 1 and \verb|NODES|. The
\verb|COST| field contains the cost of edge $(i,j)$.

\para{Example.} Below here is an example of the data file in DIMACS
format corresponding to the assignment problem shown on Fig~3.

\begin{footnotesize}
\begin{verbatim}
c sample.asn
c
c This is an example of the assignment problem data
c in DIMACS format.
c
p asn 17 22
c
n 1
n 2
n 3
n 4
n 5
n 6
n 7
n 8
c
a 1  9 13
a 1 10 21
a 1 12 20
a 2 10 12
a 2 12  8
a 2 13 26
a 3 11 22
a 3 13 11
a 4  9 12
a 4 12 36
a 4 14 25
a 5 11 41
a 5 12 40
a 5 13 11
a 5 14  4
a 5 15  8
a 5 16 35
a 5 17 32
a 6  9 13
a 7 10 19
a 8 10 39
a 8 11 15
c
c eof
\end{verbatim}
\end{footnotesize}

\newpage

\subsection{glp\_write\_asnprob --- write assignment problem data in
DIMACS format}

\synopsis

\begin{verbatim}
   int glp_write_asnprob(glp_graph *G, int v_set, int a_cost, const char *fname);
\end{verbatim}

\description

The routine \verb|glp_write_asnprob| writes the assignment problem data
to a text file in DIMACS format.

The parameter \verb|G| is the graph program object, which specifies the
assignment problem instance.

The parameter \verb|v_set| specifies an offset of the field of type
\verb|int| in the vertex data block, which contains the node set
indicator:

0 --- the node is in set $R$;

1 --- the node is in set $S$.

\noindent
If \verb|v_set| $<0$, it is assumed that a node having no incoming arcs
is in set $R$, and a node having no outgoing arcs is in set $S$.

The parameter \verb|a_cost| specifies an offset of the field of type
\verb|double| in the arc data block, which contains $c_{ij}$, the edge
cost. If \verb|a_cost| $<0$, it is assumed that $c_{ij}=1$ for all
edges.

The character string \verb|fname| specifies a name of the text file to
be written out. (If the file name ends with suffix `\verb|.gz|', the
file is assumed to be compressed, in which case the routine performs
automatic compression on writing it.)

\para{Note}

The routine \verb|glp_write_asnprob| does not check that the specified
graph object correctly represents a bipartite graph. To make sure that
the problem data are correct, use the routine \verb|glp_check_asnprob|.

\returns

If the operation was successful, the routine returns zero. Otherwise,
it prints an error message and returns non-zero.

\vspace*{-4pt}

\subsection{glp\_check\_asnprob --- check correctness of assignment
problem data}

\synopsis

\begin{verbatim}
   int glp_check_asnprob(glp_graph *G, int v_set);
\end{verbatim}

\description

The routine \verb|glp_check_asnprob| checks that the specified graph
object \verb|G| correctly represents a bipartite graph.

The parameter \verb|v_set| specifies an offset of the field of type
\verb|int| in the vertex data block, which contains the node set
indicator:

0 --- the node is in set $R$;

1 --- the node is in set $S$.

\noindent
If \verb|v_set| $<0$, it is assumed that a node having no incoming arcs
is in set $R$, and a node having no outgoing arcs is in set $S$.

\returns

0 --- the data are correct;

1 --- the set indicator of some node is 0, however, that node has one
or more incoming arcs;

2 --- the set indicator of some node is 1, however, that node has one
or more outgoing arcs;

3 --- the set indicator of some node is invalid (neither 0 nor 1);

4 --- some node has both incoming and outgoing arcs.

\subsection{glp\_asnprob\_lp --- convert assignment problem to LP}

\synopsis

\begin{verbatim}
   int glp_asnprob_lp(glp_prob *P, int form, glp_graph *G, int names, int v_set,
                      int a_cost);
\end{verbatim}

\description

The routine \verb|glp_asnprob_lp| builds LP problem, which corresponds
to the specified assignment problem.

The parameter \verb|P| is the resultant LP problem object to be built.
Note that on entry its current content is erased with the routine
\verb|glp_erase_prob|.

The parameter \verb|form| defines which LP formulation should be used:

\verb|GLP_ASN_MIN| --- perfect matching (15)---(18), minimization;

\verb|GLP_ASN_MAX| --- perfect matching (15)---(18), maximization;

\verb|GLP_ASN_MMP| --- maximum weighted matching (11)---(14).

The parameter \verb|G| is the graph program object, which specifies the
assignment problem instance.

The parameter \verb|names| is a flag. If it is \verb|GLP_ON|, the
routine uses symbolic names of the graph object components to assign
symbolic names to the LP problem object components. If the \verb|flag|
is \verb|GLP_OFF|, no symbolic names are assigned.

The parameter \verb|v_set| specifies an offset of the field of type
\verb|int| in the vertex data block, which contains the node set
indicator:

0 --- the node is in set $R$;

1 --- the node is in set $S$.

\noindent
If \verb|v_set| $<0$, it is assumed that a node having no incoming arcs
is in set $R$, and a node having no outgoing arcs is in set $S$.

The parameter \verb|a_cost| specifies an offset of the field of type
\verb|double| in the arc data block, which contains $c_{ij}$, the edge
cost. If \verb|a_cost| $<0$, it is assumed that $c_{ij}=1$ for all
edges.

\newpage

\returns

If the LP problem has been successfully built, the routine
\verb|glp_asnprob_lp| returns zero, otherwise, non-zero (see the
routine \verb|glp_check_asnprob|).

\para{Example}

The example program below reads the assignment problem instance in
DIMACS format from file `\verb|sample.asn|', converts the instance to
LP (11)---(14), and writes the resultant LP in CPLEX format to file
`\verb|matching.lp|'.

\begin{footnotesize}
\begin{verbatim}
#include <stddef.h>
#include <glpk.h>

typedef struct { int set; } v_data;
typedef struct { double cost; } a_data;

int main(void)
{     glp_graph *G;
      glp_prob *P;
      G = glp_create_graph(sizeof(v_data), sizeof(a_data));
      glp_read_asnprob(G, offsetof(v_data, set),
         offsetof(a_data, cost), "sample.asn");
      P = glp_create_prob();
      glp_asnprob_lp(P, GLP_ASN_MMP, G, GLP_ON,
         offsetof(v_data, set), offsetof(a_data, cost));
      glp_delete_graph(G);
      glp_write_lp(P, NULL, "matching.lp");
      glp_delete_prob(P);
      return 0;
}
\end{verbatim}
\end{footnotesize}

If `\verb|sample.asn|' is the example data file from the subsection
describing \verb|glp_read_asnprob|, file `\verb|matching.lp|' may look
like follows:

\begin{footnotesize}
\begin{verbatim}
Maximize
 obj: + 20 x(1,12) + 21 x(1,10) + 13 x(1,9) + 26 x(2,13) + 8 x(2,12)
 + 12 x(2,10) + 11 x(3,13) + 22 x(3,11) + 25 x(4,14) + 36 x(4,12)
 + 12 x(4,9) + 32 x(5,17) + 35 x(5,16) + 8 x(5,15) + 4 x(5,14)
 + 11 x(5,13) + 40 x(5,12) + 41 x(5,11) + 13 x(6,9) + 19 x(7,10)
 + 15 x(8,11) + 39 x(8,10)

Subject To
 r_1: + x(1,9) + x(1,10) + x(1,12) <= 1
 r_2: + x(2,10) + x(2,12) + x(2,13) <= 1
 r_3: + x(3,11) + x(3,13) <= 1
 r_4: + x(4,9) + x(4,12) + x(4,14) <= 1
 r_5: + x(5,11) + x(5,12) + x(5,13) + x(5,14) + x(5,15) + x(5,16)
 + x(5,17) <= 1
 r_6: + x(6,9) <= 1
 r_7: + x(7,10) <= 1
 r_8: + x(8,10) + x(8,11) <= 1
 r_9: + x(6,9) + x(4,9) + x(1,9) <= 1
 r_10: + x(8,10) + x(7,10) + x(2,10) + x(1,10) <= 1
 r_11: + x(8,11) + x(5,11) + x(3,11) <= 1
 r_12: + x(5,12) + x(4,12) + x(2,12) + x(1,12) <= 1
 r_13: + x(5,13) + x(3,13) + x(2,13) <= 1
 r_14: + x(5,14) + x(4,14) <= 1
 r_15: + x(5,15) <= 1
 r_16: + x(5,16) <= 1
 r_17: + x(5,17) <= 1

Bounds
 0 <= x(1,12) <= 1
 0 <= x(1,10) <= 1
 0 <= x(1,9) <= 1
 0 <= x(2,13) <= 1
 0 <= x(2,12) <= 1
 0 <= x(2,10) <= 1
 0 <= x(3,13) <= 1
 0 <= x(3,11) <= 1
 0 <= x(4,14) <= 1
 0 <= x(4,12) <= 1
 0 <= x(4,9) <= 1
 0 <= x(5,17) <= 1
 0 <= x(5,16) <= 1
 0 <= x(5,15) <= 1
 0 <= x(5,14) <= 1
 0 <= x(5,13) <= 1
 0 <= x(5,12) <= 1
 0 <= x(5,11) <= 1
 0 <= x(6,9) <= 1
 0 <= x(7,10) <= 1
 0 <= x(8,11) <= 1
 0 <= x(8,10) <= 1

End
\end{verbatim}
\end{footnotesize}

\subsection{glp\_asnprob\_okalg --- solve assignment problem with
out-of-kilter\\algorithm}

\synopsis

\begin{verbatim}
   int glp_asnprob_okalg(int form, glp_graph *G, int v_set, int a_cost,
                         double *sol, int a_x);
\end{verbatim}

\description

The routine \verb|glp_mincost_okalg| finds optimal solution to the
assignment problem with the out-of-kilter
algorithm.\footnote{GLPK implementation of the out-of-kilter algorithm
is based on the following book: L.~R.~Ford,~Jr., and D.~R.~Fulkerson,
``Flows in Networks,'' The RAND Corp., Report R-375-PR (August 1962),
Chap. III ``Minimal Cost Flow Problems,'' pp.~113-26.} Note that this
routine requires all the problem data to be integer-valued.

The parameter \verb|form| defines which LP formulation should be used:

\verb|GLP_ASN_MIN| --- perfect matching (15)---(18), minimization;

\verb|GLP_ASN_MAX| --- perfect matching (15)---(18), maximization;

\verb|GLP_ASN_MMP| --- maximum weighted matching (11)---(14).

\newpage

The parameter \verb|G| is the graph program object, which specifies the
assignment problem instance.

The parameter \verb|v_set| specifies an offset of the field of type
\verb|int| in the vertex data block, which contains the node set
indicator:

0 --- the node is in set $R$;

1 --- the node is in set $S$.

\noindent
If \verb|v_set| $<0$, it is assumed that a node having no incoming arcs
is in set $R$, and a node having no outgoing arcs is in set $S$.

The parameter \verb|a_cost| specifies an offset of the field of type
\verb|double| in the arc data block, which contains $c_{ij}$, the edge
cost. This value must be integer in the range [\verb|-INT_MAX|,
\verb|+INT_MAX|]. If \verb|a_cost| $<0$, it is assumed that $c_{ij}=1$
for all edges.

The parameter \verb|sol| specifies a location, to which the routine
stores the objective value (that is, the total cost) found.
If \verb|sol| is \verb|NULL|, the objective value is not stored.

The parameter \verb|a_x| specifies an offset of the field of type
\verb|int| in the arc data block, to which the routine stores $x_{ij}$.
If \verb|a_x| $<0$, this value is not stored.

\returns

\begin{retlist}
0 & Optimal solution found.\\

\verb|GLP_ENOPFS| & No (primal) feasible solution exists.\\

\verb|GLP_EDATA| & Unable to start the search, because the assignment
problem data are either incorrect (this error is detected by the
routine \verb|glp_check_asnprob|), not integer-valued or out of range.\\

\verb|GLP_ERANGE| & The search was prematurely terminated because of
integer overflow.\\

\verb|GLP_EFAIL| & An error has been detected in the program logic.
(If this code is returned for your problem instance, please report to
\verb|<bug-glpk@gnu.org>|.)\\
\end{retlist}

\para{Comments}

Since the out-of-kilter algorithm is designed to find a minimal cost
circulation, the routine \verb|glp_asnprob_okalg| converts the original
graph to a network suitable for this algorithm in the following
way:\footnote{The conversion is performed internally and does not
change the original graph program object passed to the routine.}

1) it replaces each edge $(i,j)$ by arc $(i\rightarrow j)$,
flow $x_{ij}$ through which has zero lower bound ($l_{ij}=0$), unity
upper bound ($u_{ij}=1$), and per-unit cost $+c_{ij}$ (in case of
\verb|GLP_ASN_MIN|), or $-c_{ij}$ (in case of \verb|GLP_ASN_MAX| and
\verb|GLP_ASN_MMP|);

2) then it adds one auxiliary feedback node $k$;

3) for each original node $i\in R$ the routine adds auxiliary supply
arc $(k\rightarrow i)$, flow $x_{ki}$ through which is costless
($c_{ki}=0$) and either fixed at 1 ($l_{ki}=u_{ki}=1$, in case of
\verb|GLP_ASN_MIN| and \verb|GLP_ASN_MAX|) or has zero lower bound and
unity upper bound ($l_{ij}=0$, $u_{ij}=1$, in case of
\verb|GLP_ASN_MMP|);

\newpage

4) similarly, for each original node $j\in S$ the routine adds
auxiliary demand arc $(j\rightarrow k)$, flow $x_{jk}$ through which is
costless ($c_{jk}=0$) and either fixed at 1 ($l_{jk}=u_{jk}=1$, in case
of \verb|GLP_ASN_MIN| and \verb|GLP_ASN_MAX|) or has zero lower bound
and unity upper bound ($l_{jk}=0$, $u_{jk}=1$, in case of
\verb|GLP_ASN_MMP|).

\para{Example}

The example program shown below reads the assignment problem instance
in DIMACS format from file `\verb|sample.asn|', solves it by using the
routine \verb|glp_asnprob_okalg|, and writes the solution found to the
standard output.

\begin{footnotesize}
\begin{verbatim}
#include <stddef.h>
#include <stdio.h>
#include <stdlib.h>
#include <glpk.h>

typedef struct { int set; } v_data;
typedef struct { double cost; int x; } e_data;

#define node(v) ((v_data *)((v)->data))
#define edge(e) ((e_data *)((e)->data))

int main(void)
{     glp_graph *G;
      glp_vertex *v;
      glp_arc *e;
      int i, ret;
      double sol;
      G = glp_create_graph(sizeof(v_data), sizeof(e_data));
      glp_read_asnprob(G, offsetof(v_data, set),
         offsetof(e_data, cost), "sample.asn");
      ret = glp_asnprob_okalg(GLP_ASN_MMP, G,
         offsetof(v_data, set), offsetof(e_data, cost), &sol,
         offsetof(e_data, x));
      printf("ret = %d; sol = %5g\n", ret, sol);
      for (i = 1; i <= G->nv; i++)
      {  v = G->v[i];
         for (e = v->out; e != NULL; e = e->t_next)
            printf("edge %2d %2d: x = %d; c = %g\n",
               e->tail->i, e->head->i, edge(e)->x, edge(e)->cost);
      }
      glp_delete_graph(G);
      return 0;
}
\end{verbatim}
\end{footnotesize}

If `\verb|sample.asn|' is the example data file from the subsection
describing \verb|glp_read_asnprob|, the output may look like follows:

\begin{footnotesize}
\begin{verbatim}
Reading assignment problem data from `sample.asn'...
Assignment problem has 8 + 9 = 17 nodes and 22 arcs
38 lines were read
ret = 0; sol =   180
edge  1 12: x = 1; c = 20
edge  1 10: x = 0; c = 21
edge  1  9: x = 0; c = 13
edge  2 13: x = 1; c = 26
edge  2 12: x = 0; c = 8
edge  2 10: x = 0; c = 12
edge  3 13: x = 0; c = 11
edge  3 11: x = 1; c = 22
edge  4 14: x = 1; c = 25
edge  4 12: x = 0; c = 36
edge  4  9: x = 0; c = 12
edge  5 17: x = 0; c = 32
edge  5 16: x = 1; c = 35
edge  5 15: x = 0; c = 8
edge  5 14: x = 0; c = 4
edge  5 13: x = 0; c = 11
edge  5 12: x = 0; c = 40
edge  5 11: x = 0; c = 41
edge  6  9: x = 1; c = 13
edge  7 10: x = 0; c = 19
edge  8 11: x = 0; c = 15
edge  8 10: x = 1; c = 39
\end{verbatim}
\end{footnotesize}

\subsection{glp\_asnprob\_hall --- find bipartite matching of maximum
cardinality}

\synopsis

\begin{verbatim}
   int glp_asnprob_hall(glp_graph *G, int v_set, int a_x);
\end{verbatim}

\description

The routine \verb|glp_asnprob_hall| finds a matching of maximal
cardinality in the specified bipartite graph. It uses a version of the
Fortran routine \verb|MC21A| developed by
I.~S.~Duff\footnote{I.~S.~Duff, Algorithm 575: Permutations for
zero-free diagonal, ACM Trans. on Math. Softw. 7 (1981),\linebreak
pp.~387-390.}, which implements Hall's algorithm.\footnote{M.~Hall,
``An Algorithm for Distinct Representatives,'' Am. Math. Monthly 63
(1956), pp.~716-717.}

The parameter \verb|G| is a pointer to the graph program object.

The parameter \verb|v_set| specifies an offset of the field of type
\verb|int| in the vertex data block, which contains the node set
indicator:

0 --- the node is in set $R$;

1 --- the node is in set $S$.

\noindent
If \verb|v_set| $<0$, it is assumed that a node having no incoming arcs
is in set $R$, and a node having no outgoing arcs is in set $S$.

The parameter \verb|a_x| specifies an offset of the field of type
\verb|int| in the arc data block, to which the routine stores $x_{ij}$.
If \verb|a_x| $<0$, this value is not stored.

\returns

The routine \verb|glp_asnprob_hall| returns the cardinality of the
matching found. However, if the specified graph is incorrect (as
detected by the routine \verb|glp_check_asnprob|), this routine returns
a negative value.

\newpage

\para{Comments}

The same solution may be obtained with the routine
\verb|glp_asnprob_okalg| (for LP formulation \verb|GLP_ASN_MMP| and
all edge costs equal to 1). However, the routine
\verb|glp_asnprob_hall| is much faster.

\para{Example}

The example program shown below reads the assignment problem instance
in DIMACS format from file `\verb|sample.asn|', finds a bipartite
matching of maximal cardinality by using the routine
\verb|glp_asnprob_hall|, and writes the solution found to the standard
output.

\begin{footnotesize}
\begin{verbatim}
#include <stddef.h>
#include <stdio.h>
#include <stdlib.h>
#include <glpk.h>

typedef struct { int set; } v_data;
typedef struct { int x;   } e_data;

#define node(v) ((v_data *)((v)->data))
#define edge(e) ((e_data *)((e)->data))

int main(void)
{     glp_graph *G;
      glp_vertex *v;
      glp_arc *e;
      int i, card;
      G = glp_create_graph(sizeof(v_data), sizeof(e_data));
      glp_read_asnprob(G, offsetof(v_data, set), -1,
         "sample.asn");
      card = glp_asnprob_hall(G, offsetof(v_data, set),
         offsetof(e_data, x));
      printf("card = %d\n", card);
      for (i = 1; i <= G->nv; i++)
      {  v = G->v[i];
         for (e = v->out; e != NULL; e = e->t_next)
            printf("edge %2d %2d: x = %d\n",
               e->tail->i, e->head->i, edge(e)->x);
      }
      glp_delete_graph(G);
      return 0;
}
\end{verbatim}
\end{footnotesize}

If `\verb|sample.asn|' is the example data file from the subsection
describing \verb|glp_read_asnprob|, the output may look like follows:

\begin{footnotesize}
\begin{verbatim}
Reading assignment problem data from `sample.asn'...
Assignment problem has 8 + 9 = 17 nodes and 22 arcs
38 lines were read
card = 7
edge  1 12: x = 1
edge  1 10: x = 0
edge  1  9: x = 0
edge  2 13: x = 1
edge  2 12: x = 0
edge  2 10: x = 0
edge  3 13: x = 0
edge  3 11: x = 1
edge  4 14: x = 1
edge  4 12: x = 0
edge  4  9: x = 0
edge  5 17: x = 1
edge  5 16: x = 0
edge  5 15: x = 0
edge  5 14: x = 0
edge  5 13: x = 0
edge  5 12: x = 0
edge  5 11: x = 0
edge  6  9: x = 1
edge  7 10: x = 1
edge  8 11: x = 0
edge  8 10: x = 0
\end{verbatim}
\end{footnotesize}

%%%%%%%%%%%%%%%%%%%%%%%%%%%%%%%%%%%%%%%%%%%%%%%%%%%%%%%%%%%%%%%%%%%%%%%%

\newpage

\section{Critical path problem}

\subsection{Background}

The {\it critical path problem} (CPP) is stated as follows. Let there
be given a project $J$, which is a set of jobs (tasks, activities,
etc.). Performing each job $i\in J$ requires time $t_i\geq 0$. Besides,
over the set $J$ there is given a precedence relation
$R\subseteq J\times J$, where $(i,j)\in R$ means that job $i$
immediately precedes job $j$, i.e. performing job $j$ cannot start
until job $i$ has been completely performed. The problem is to find
starting times $x_i$ for each job $i\in J$, which satisfy to the
precedence relation and minimize the total duration (makespan) of the
project.

The following is an example of the critical path problem:

\bigskip

\begin{center}
\begin{tabular}{|c|l|c|c|}
\hline
Job&Desription&Time&Predecessors\\
\hline
A&Excavate&3&---\\
B&Lay foundation&4&A\\
C&Rough plumbing&3&B\\
D&Frame&10&B\\
E&Finish exterior&8&D\\
F&Install HVAC&4&D\\
G&Rough electric&6&D\\
H&Sheet rock&8&C, E, F, G\\
I&Install cabinets&5&H\\
J&Paint&5&H\\
K&Final plumbing&4&I\\
L&Final electric&2&J\\
M&Install flooring&4&K, L\\
\hline
\end{tabular}
\end{center}

\bigskip

Obviously, the project along with the precedence relation can be
represented as a directed graph $G=(J,R)$ called {\it project network},
where each node $i\in J$ corresponds to a job, and arc
$(i\rightarrow j)\in R$ means that job $i$ immediately precedes job
$j$.\footnote{There exists another network representation of the
critical path problem, where jobs correspond to arcs while nodes
correspond to events introduced to express the precedence relation.
That representation, however, is much less convenient than the one,
where jobs are represented as nodes of the network.} The project network
for the example above is shown on Fig.~4.

May note that the project network must be acyclic; otherwise, it would
be impossible to satisfy to the precedence relation for any job that
belongs to a cycle.

\newpage

\hspace*{.5in}
\xymatrix
{&&&C|3\ar[rd]&&I|5\ar[r]&K|4\ar[rd]&\\
A|3\ar[r]&B|4\ar[rru]\ar[rd]&&E|8\ar[r]&H|8\ar[ru]\ar[rd]&&&M|4\\
&&D|10\ar[ru]\ar[r]\ar[rd]&F|4\ar[ru]&&J|5\ar[r]&L|2\ar[ru]&\\
&&&G|6\ar[ruu]&&&&\\
}

\medskip

\noindent\hfil
Fig.~4. An example of the project network.

\medskip

The critical path problem can be naturally formulated as the following
LP problem:

\medskip

\noindent
\hspace{.5in}minimize
$$z\eqno(19)$$
\hspace{.5in}subject to
$$x_i+t_i\leq z\ \ \ \hbox{for all}\ i\in J\ \ \ \ \eqno(20)$$
$$x_i+t_i\leq x_j\ \ \ \hbox{for all}\ (i,j)\in R\eqno(21)$$
$$x_i\geq 0\ \ \ \ \ \ \ \hbox{for all}\ i\in J\ \ \eqno(22)$$

The inequality constraints (21), which are active in the optimal
solution, define so called {\it critical path} having the following
property: the minimal project duration $z$ can be decreased only by
decreasing the times $t_j$ for jobs on the critical path, and delaying
any critical job delays the entire project.

\subsection{glp\_cpp --- solve critical path problem}

\synopsis

\begin{verbatim}
   double glp_cpp(glp_graph *G, int v_t, int v_es, int v_ls);
\end{verbatim}

\description

The routine \verb|glp_cpp| solves the critical path problem represented
in the form of the project network.

The parameter \verb|G| is a pointer to the graph object, which
specifies the project network. This graph must be acyclic. Multiple
arcs are allowed being considered as single arcs.

The parameter \verb|v_t| specifies an offset of the field of type
\verb|double| in the vertex data block, which contains time $t_i\geq 0$
needed to perform corresponding job $j\in J$. If \verb|v_t| $<0$, it is
assumed that $t_i=1$ for all jobs.

The parameter \verb|v_es| specifies an offset of the field of type
\verb|double| in the vertex data block, to which the routine stores
the {\it earliest start time} for corresponding job. If \verb|v_es|
$<0$, this time is not stored.

\newpage

The parameter \verb|v_ls| specifies an offset of the field of type
\verb|double| in the vertex data block, to which the routine stores
the {\it latest start time} for corresponding job. If \verb|v_ls|
$<0$, this time is not stored.

The difference between the latest and earliest start times of some job
is called its {\it time reserve}. Delaying a job within its time
reserve does not affect the project duration, so if the time reserve is
zero, the corresponding job is critical.

\para{Returns}

The routine \verb|glp_cpp| returns the minimal project duration, i.e.
minimal time needed to perform all jobs in the project.

\para{Example}

The example program below solves the critical path problem shown on
Fig.~4 by using the routine \verb|glp_cpp| and writes the solution
found on the standard output.

\begin{footnotesize}
\begin{verbatim}
#include <stddef.h>
#include <stdio.h>
#include <stdlib.h>
#include <glpk.h>

typedef struct { double t, es, ls; } v_data;

#define node(v) ((v_data *)((v)->data))

int main(void)
{     glp_graph *G;
      int i;
      double t, es, ef, ls, lf, total;
      G = glp_create_graph(sizeof(v_data), 0);
      glp_add_vertices(G, 13);
      node(G->v[1])->t = 3;   /* A: Excavate */
      node(G->v[2])->t = 4;   /* B: Lay foundation */
      node(G->v[3])->t = 3;   /* C: Rough plumbing */
      node(G->v[4])->t = 10;  /* D: Frame */
      node(G->v[5])->t = 8;   /* E: Finish exterior */
      node(G->v[6])->t = 4;   /* F: Install HVAC */
      node(G->v[7])->t = 6;   /* G: Rough elecrtic */
      node(G->v[8])->t = 8;   /* H: Sheet rock */
      node(G->v[9])->t = 5;   /* I: Install cabinets */
      node(G->v[10])->t = 5;  /* J: Paint */
      node(G->v[11])->t = 4;  /* K: Final plumbing */
      node(G->v[12])->t = 2;  /* L: Final electric */
      node(G->v[13])->t = 4;  /* M: Install flooring */
      glp_add_arc(G, 1, 2);   /* A precedes B */
      glp_add_arc(G, 2, 3);   /* B precedes C */
      glp_add_arc(G, 2, 4);   /* B precedes D */
      glp_add_arc(G, 4, 5);   /* D precedes E */
      glp_add_arc(G, 4, 6);   /* D precedes F */
      glp_add_arc(G, 4, 7);   /* D precedes G */
      glp_add_arc(G, 3, 8);   /* C precedes H */
      glp_add_arc(G, 5, 8);   /* E precedes H */
      glp_add_arc(G, 6, 8);   /* F precedes H */
      glp_add_arc(G, 7, 8);   /* G precedes H */
      glp_add_arc(G, 8, 9);   /* H precedes I */
      glp_add_arc(G, 8, 10);  /* H precedes J */
      glp_add_arc(G, 9, 11);  /* I precedes K */
      glp_add_arc(G, 10, 12); /* J precedes L */
      glp_add_arc(G, 11, 13); /* K precedes M */
      glp_add_arc(G, 12, 13); /* L precedes M */
      total = glp_cpp(G, offsetof(v_data, t), offsetof(v_data, es),
         offsetof(v_data, ls));
      printf("Minimal project duration is %.2f\n\n", total);
      printf("Job  Time      ES     EF     LS     LF\n");
      printf("--- ------   ------ ------ ------ ------\n");
      for (i = 1; i <= G->nv; i++)
      {  t = node(G->v[i])->t;
         es = node(G->v[i])->es;
         ef = es + node(G->v[i])->t;
         ls = node(G->v[i])->ls;
         lf = ls + node(G->v[i])->t;
         printf("%3d %6.2f %s %6.2f %6.2f %6.2f %6.2f\n",
            i, t, ls - es < 0.001 ? "*" : " ", es, ef, ls, lf);
      }
      glp_delete_graph(G);
      return 0;
}
\end{verbatim}
\end{footnotesize}

The output from the example program shown below includes job number,
the time needed to perform a job, earliest start time (\verb|ES|),
earliest finish time (\verb|EF|), latest start time (\verb|LS|), and
latest finish time (\verb|LF|) for each job in the project. Critical
jobs are marked by asterisks.

\begin{footnotesize}
\begin{verbatim}
Minimal project duration is 46.00

Job  Time      ES     EF     LS     LF
--- ------   ------ ------ ------ ------
  1   3.00 *   0.00   3.00   0.00   3.00
  2   4.00 *   3.00   7.00   3.00   7.00
  3   3.00     7.00  10.00  22.00  25.00
  4  10.00 *   7.00  17.00   7.00  17.00
  5   8.00 *  17.00  25.00  17.00  25.00
  6   4.00    17.00  21.00  21.00  25.00
  7   6.00    17.00  23.00  19.00  25.00
  8   8.00 *  25.00  33.00  25.00  33.00
  9   5.00 *  33.00  38.00  33.00  38.00
 10   5.00    33.00  38.00  35.00  40.00
 11   4.00 *  38.00  42.00  38.00  42.00
 12   2.00    38.00  40.00  40.00  42.00
 13   4.00 *  42.00  46.00  42.00  46.00
\end{verbatim}
\end{footnotesize}

%%%%%%%%%%%%%%%%%%%%%%%%%%%%%%%%%%%%%%%%%%%%%%%%%%%%%%%%%%%%%%%%%%%%%%%%

\chapter{Graph Optimization API Routines}

\section{Maximum clique problem}

\subsection{Background}

The {\it maximum clique problem (MCP)} is a classic combinatorial
optimization problem. Given an undirected graph $G=(V,E)$, where $V$ is
a set of vertices, and $E$ is a set of edges, this problem is to find
the largest {\it clique} $C\subseteq G$, i.e. the largest induced
complete subgraph. A generalization of this problem is the {\it maximum
weight clique problem (MWCP)}, which is to find a clique $C\subseteq G$
of the largest weight $\displaystyle\sum_{v\in C}w(v)\rightarrow\max$,
where $w(v)$ is a weight of vertex $v\in V$.

An example of the maximum weight clique problem is shown on Fig.~5.

\begin{figure}
\noindent\hfil
\begin{tabular}{c}
{\xymatrix %@C=16pt
{&&&{v_1}\ar@{-}[lllddd]\ar@{-}[llddddd]\ar@{-}[dddddd]
\ar@{-}[rrrddd]&&&\\
&{v_2}\ar@{-}[rrrr]\ar@{-}[rrrrdddd]\ar@{-}[rrddddd]\ar@{-}[dddd]&&&&
{v_3}\ar@{-}[llllldd]\ar@{-}[lllldddd]\ar@{-}[dddd]&\\
&&&&&&\\
{v_4}\ar@{-}[rrrrrr]\ar@{-}[rrrddd]&&&&&&{v_5}\ar@{-}[lllddd]
\ar@{-}[ldd]\\
&&&&&&\\
&{v_6}\ar@{-}[rrrr]&&&&{v_7}&\\
&&&{v_8}&&&\\
}}
\end{tabular}
\begin{tabular}{r@{\ }c@{\ }l}
$w(v_1)$&=&3\\$w(v_2)$&=&4\\$w(v_3)$&=&8\\$w(v_4)$&=&1\\
$w(v_5)$&=&5\\$w(v_6)$&=&2\\$w(v_7)$&=&1\\$w(v_8)$&=&3\\
\end{tabular}

\bigskip

\begin{center}
Fig.~5. An example of the maximum weight clique problem.
\end{center}
\end{figure}

\subsection{glp\_wclique\_exact --- find maximum weight clique with
exact algorithm}

\synopsis

\begin{verbatim}
   int glp_wclique_exact(glp_graph *G, int v_wgt, double *sol, int v_set);
\end{verbatim}

\description

The routine {\tt glp\_wclique\_exact} finds a maximum weight clique in
the specified undirected graph with the exact algorithm developed by
Patric \"Osterg{\aa}rd.\footnote{P.~R.~J.~\"Osterg{\aa}rd, A new
algorithm for the maximum-weight clique problem, Nordic J. of
Computing, Vol.~8, No.~4, 2001, pp.~424--36.}

The parameter {\tt G} is the program object, which specifies
an undirected graph. Each arc $(x\rightarrow y)$ in {\tt G} is
considered as edge $(x,y)$, self-loops are ignored, and multiple edges,
if present, are replaced (internally) by simple edges.

The parameter {\tt v\_wgt} specifies an offset of the field of type
{\tt double} in the vertex data block, which contains a weight of
corresponding vertex. Vertex weights must be integer-valued in the
range $[0,$ {\tt INT\_MAX}$]$. If {\tt v\_wgt} $<0$, it is assumed that
all vertices of the graph have the weight 1.

\newpage

The parameter {\tt sol} specifies a location, to which the routine
stores the weight of the clique found (the clique weight is the sum
of weights of all vertices included in the clique.) If {\tt sol} is
{\tt NULL}, the solution is not stored.

The parameter {\tt v\_set} specifies an offset of the field of type
{\tt int} in the vertex data block, to which the routines stores a
vertex flag: 1 means that the corresponding vertex is included in the
clique found, and 0 otherwise. If {\tt v\_set} $<0$, vertex flags are
not stored.

\returns

\begin{retlist}
0 & Optimal solution found.\\

\verb|GLP_EDATA| & Unable to start the search, because some vertex
weights are either not integer-valued or out of range. This code is
also returned if the sum of weights of all vertices exceeds
{\tt INT\_MAX}. \\
\end{retlist}

\para{Notes}

1. The routine {\it glp\_wclique\_exact} finds exact solution. Since
both MCP and MWCP problems are NP-complete, the algorithm may require
exponential time in worst cases.

2. Internally the specified graph is converted to an adjacency matrix
in {\it dense} format. This requires about $|V|^2/16$ bytes of memory,
where $|V|$ is the number of vertices in the graph.

\para{Example}

The example program shown below reads a MWCP instance in DIMACS
clique/coloring format from file `\verb|sample.clq|', finds the clique
of largest weight, and writes the solution found on the standard
output.

\newpage

\begin{footnotesize}
\begin{verbatim}
#include <stddef.h>
#include <stdio.h>
#include <stdlib.h>
#include <glpk.h>

typedef struct { double wgt; int set; } v_data;

#define vertex(v) ((v_data *)((v)->data))

int main(void)
{     glp_graph *G;
      glp_vertex *v;
      int i, ret;
      double sol;
      G = glp_create_graph(sizeof(v_data), 0);
      glp_read_ccdata(G, offsetof(v_data, wgt), "sample.clq");
      ret = glp_wclique_exact(G, offsetof(v_data, wgt), &sol,
         offsetof(v_data, set));
      printf("ret = %d; sol = %g\n", ret, sol);
      for (i = 1; i <= G->nv; i++)
      {  v = G->v[i];
         printf("vertex %d: weight = %g, flag = %d\n",
            i, vertex(v)->wgt, vertex(v)->set);
      }
      glp_delete_graph(G);
      return 0;
}
\end{verbatim}
\end{footnotesize}

For the example shown on Fig.~5 the data file may look like follows:

\begin{footnotesize}
\begin{verbatim}
c sample.clq
c
c This is an example of the maximum weight clique
c problem in DIMACS clique/coloring format.
c
p edge 8 16
n 1 3
n 2 4
n 3 8
n 5 5
n 6 2
n 8 3
e 1 4
e 1 5
e 1 6
e 1 8
e 2 3
e 2 6
e 2 7
e 2 8
e 3 4
e 3 6
e 3 7
e 4 5
e 4 8
e 5 7
e 5 8
e 6 7
c
c eof
\end{verbatim}
\end{footnotesize}

The corresponding output from the example program is the following:

\begin{footnotesize}
\begin{verbatim}
Reading graph from `sample.clq'...
Graph has 8 vertices and 16 edges
28 lines were read
ret = 0; sol = 15
vertex 1: weight = 3, flag = 0
vertex 2: weight = 4, flag = 1
vertex 3: weight = 8, flag = 1
vertex 4: weight = 1, flag = 0
vertex 5: weight = 5, flag = 0
vertex 6: weight = 2, flag = 1
vertex 7: weight = 1, flag = 1
vertex 8: weight = 3, flag = 0
\end{verbatim}
\end{footnotesize}

\end{document}
